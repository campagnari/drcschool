% !TEX spellcheck = de_DE
\documentclass[hyperworksheet]{drcschool}
% Die Klassenoption `nofonts` beeinflusst die Schriftart nicht.
%                   `notikz` lädt TikZ nicht.
%                   `nopgfplots` lädt pgfplots nicht.
%                   `hyperworksheet` lädt hyperref und definiert ausfüllbare Arbeitsblätter.
% Alle andere Optionen werden an die zu Grunde liegende Klasse scrartcl weitergegeben.

% Titel und Klasse sind notwendig für die {schedule} Umgebung
\lesson{Thema}[Möglicher Untertitel]
\class{8g}

% Datum ist eigentlich nur für Klassenarbeit relevant
\date{30.\ Februar 1712}

%%%%%%%%%%%%%%%%%%%%%%%%%%%%%%%%%%%%%%%%%%%%%%%%%%%%%%%%%%%%%%%%%%%%%%%%%%%%
%   !! Dieser Teil ist NUR für diese Dokumentation !!                      %
\makeatletter                                                              %
\newcommand*{\bcode}{%                                                     %
   \vskip.5ex\begingroup                                                   %
   \parindent\z@\advance\leftskip4em\ttfamily                              %
   \def\\{\textbackslash}\def\nl{\newline}%                                %
}                                                                          %
\newcommand*{\ecode}{\par\endgroup\vskip.5ex\@doendpe}                     %
\newcommand*{\cs}[1]{\textup{\ttfamily\textbackslash#1}}                   %
\newcommand*{\str}[1]{\textup{\ttfamily\string#1}}                         %
\newcommand*{\marg}[1]{\textup{\ttfamily\{#1\}}}                           %
\newcommand*{\oarg}[1]{\textup{\ttfamily[#1]}}                             %
\newcommand*{\pkg}[1]{\textup{\ttfamily#1}}                                %
\newcommand*{\opt}[1]{\textup{\ttfamily#1}}                                %
\newcommand*{\env}[1]{\textup{\ttfamily\{#1\}}}                            %
\newcommand*{\param}[1]{\mbox{\normalfont$\langle$\textit{#1}$\rangle$}}   %
\newcommand*{\optstar}{\mbox{\ttfamily$\langle$*$\rangle$}}                %
\newcommand*{\TikZ}{\textup{Ti\textit{k}Z}}                                %
\newcommand*{\onlytikz}{%                                                  %
   \textcolor{red}{%                                                       %
      \vtop{%                                                              %
         \vbox{%                                                           %
            \vss                                                           %
            \hrule height\p@                                               %
            \kern2\p@                                                      %
            \hbox{\normalfont\normalsize! \TikZ\ !}%                       %
         }%                                                                %
         \kern2\p@                                                         %
         \hrule height\p@                                                  %
      }%                                                                   %
   }%                                                                      %
}                                                                          %
\makeatother                                                               %
\usepackage{lipsum}                                                        %
\usepackage{embedfile}                                                     %
   \embedfile{\jobname.tex}                                                %
   \embedfile{drcschool.cls}                                               %
   \embedfile{drcschool.cwl}                                               %
%%%%%%%%%%%%%%%%%%%%%%%%%%%%%%%%%%%%%%%%%%%%%%%%%%%%%%%%%%%%%%%%%%%%%%%%%%%%

\begin{document}

\begin{center}
\Large\bfseries Die Klasse \texttt{drcschool} (\csname ver@drcschool.cls\endcsname)\\
\normalfont\normalsize Davide Campagnari\\
\small\today
\end{center}
Mehr als eine Dokumentation ist dies eine Sammlung von Beispielen, die
(hoffentlich) alle wesentlichen Fähigkeiten der Klasse \pkg{drcschool} zeigen.
Sie sind allerdings nicht vollständig und es gibt in der Tat einige Features,
die hier nicht beschrieben werden. Der Grund dafür ist, dass sie meiner Meinung
nach noch nicht ganz reif sind und ich sie daher als noch nicht ,,offiziell``
betrachte. Wer durch den Code gehen will, um diese zu entdecken, soll gewarnt
sein, dass nicht beschriebene Features geändert oder gar gestrichen werden
könnten\ldots

\minisec{Eingebettete Dateien}
In dieser PDF-Datei sind folgende Dateien eingebettet:
\begin{compactdesc}
\item[\ttfamily drcschool.cls] Die Klasse selbst.
\item[\ttfamily drcschool\textunderscore template.tex] Der Quellcode dieser Dokumentation.
\item[\ttfamily drcschool.cwl] Die \opt{cwl}-Datei für \TeX Studio, damit sich der Editor
nicht ständig über nicht-definierte Befehle ärgert. Das muss natürlich an der
richtigen Stelle platziert werden: Unter Windows~10 in
\bcode
C:\\Users\\\param{User}\\AppData\\Roaming\\texstudio\\completion\\user
\ecode
während unter Linux bei
\bcode
\textasciitilde/.config/texstudio/completion/user/
\ecode
(Aber dafür gebe ich keine Garantie.)
Ob (und gegebenenfalls wie) das für andere
\LaTeX-Editoren geht, weiß ich ehrlich gesagt nicht. Eine mögliche Lösung für
\TeX works kann unter
\bcode
\url{https://tex.stackexchange.com/q/118038}
\ecode
gefunden werden, ist aber relativ alt (2013). Keine Gewähr!
\end{compactdesc}

\section{Laden der Klasse und Optionen}
Die Klasse wird ganz normal mittels
\bcode
\\documentclass[\param{Optionen}]\{drcschool\}
\ecode
geladen. Zu Grunde liegt die \KOMAScript-Klasse \pkg{scrartcl}, deren
entsprechende Optionen als Optionen zu \pkg{drcschool} weitergegeben werden können.

Es gibt einige wenige eigene Optionen:
\begin{compactdesc}
\item[\opt{nofonts}] Dies hebt die Standardeinstellung für die
   Schriftart aus; standardmäßig wird eine serifenlose Schrift gewählt.\footnote{%
      Für die Neugierigen: \pkg{tgheros} mit \pkg{newtxsf}, plus einige persönliche Vorlieben.}
   Der aktuelle Stand der Forschung scheint zu sein, dass Kinder mit LRS serifenlose
   Schriften besser finden. (Zugegeben, für Kinder \emph{ohne} LRS ist es andersrum\ldots)
   Soll jemand meine Standardeinstellung nicht mögen,\footnote{%
      ,,\texttt{drc}`` ist eine Abkürzung für meinen Namen (,,\texttt{r}`` steht hierbei
      für meinen zweiten Vornamen, nicht für den Doktortitel --- auf se einem Egotrip bin ich nicht\ldots).
      Also ja, ich gestalte die Klasse so, dass sie \emph{mir} gefällt.
      Dass andere sie nützlich finden, ist nur ein netter Nebeneffekt~\texttt{;-)}}
   so kann die Klasse mit der Option \opt{nofonts} geladen werden; daraufhin kann jede/r
   sämtliche Lieblingsschriftarten auf gewohnte Weise laden.
\item[\opt{notikz}] Einige Features der Klasse hängen von \TikZ\ ab. Das bedeutet allerdings,
   dass das Kompilieren einer einfachen ,,Hello world`` Datei verhältnismäßig lange dauern kann,
   denn \TikZ\ ist ein ziemlich großes Paket. Mit dieser Option kann das Laden von \TikZ\ verhindert werden.
\item[\opt{nopgfplots}] Das Gleiche gilt ähnlich für \pkg{pgfplots}. Will man die \TikZ-Fähigkeiten
   haben aber braucht man keine Funktionen plotten, so kann man diese Option verwenden
   und sich das Laden von \pkg{pgfplots} ersparen.
\item[\opt{hyperworksheet}] Aktiviert die Umgebung für ausfüllbare PDF-Dateien (s.~Abschnitt~\ref{sec:hy}).
\end{compactdesc}

\section{Abhängigkeiten}
Hier folgt eine kurze Liste (ohne Abhängigkeiten) der geladenen Pakete:\par\medskip\noindent
\begin{compactdesc}
\item[Schriften] \pkg{fontenc} mit Option \opt{T1}, \pkg{microtype}, \pkg{tgheros} und \pkg{newtxsf}.
   Diese werden nicht geladen
   \begin{compactitem}
   \item wenn die Klassenoption \opt{nofonts} verwendet wird (bzw.~wenn man
   \verb=\NoFonts= in einer Konfigurationsdatei verwendet, siehe
   Abschnitt~\ref{sec:anpassungen}), oder
   \item wenn man Lua\TeX\ oder X\lower.5ex\hbox{\reflectbox{E}}\kern-.16em\TeX\
         verwendet.
   \end{compactitem}
   Einige Symbole werden aus \pkg{fontawesome} genommen
   (jedoch nur einzeln importiert, die Pakete selbst werden nicht geladen).
\item[Sprache] \pkg{babel} mit Option \opt{ngerman}.
\item[Tabellen] \pkg{tabularx}, \pkg{booktabs}, \pkg{colortbl}, \pkg{longtable}.
\item[Seitenmanagement] \pkg{scrlayer-scrpage}, \pkg{geometry}.
\item[Grafik] \pkg{pict2e}.
\item[Verschiedenes] \pkg{amsgen}, \pkg{environ}\footnote{%
   Seit 2018 bietet \pkg{xparse} ein \opt{b}-Argument für Umgebungen, die den gesamten Inhalt
   sammeln sollen; seit 2020 ist \pkg{xparse} außerdem im \LaTeX-Kernel integriert. Da aber
   auf meinem Büro-Rechner noch TeX Live~2017 läuft, muss ich bei \pkg{environ} bleiben.},
   \pkg{paralist}, \pkg{icomma}, \pkg{tikz},\footnote{Es sei denn, die Klasse wurde mit der Option \opt{notikz} geladen.}
    \pkg{pgfplots}.\footnote{Es sei denn, die Klasse wurde mit der Option \opt{nopgfplots} oder \opt{notikz} geladen.}
\end{compactdesc}
Die Verwendung von \pkg{geometry} mit einer \KOMAScript-Klasse ist etwas
ungewöhnlich, war aber aus \TeX nischen Gründen nötig. Dies bedeutet allerdings,
dass die Fähigkeiten von \pkg{typearea} nicht wirklich verwendet werden können.

\begin{description}
\item[Bemerkung]
Das geübte Auge wird sehen, dass ich Einiges ermöglicht habe, was eigentlich
nicht der normalen \LaTeX-Syntax entspricht.
% Es gibt beispielsweise Umgebungen, die auch
%als Makro mit Argumenten funktionieren können.
Dies ist \emph{sehr schlechter} \TeX-Stil,
ist aber aus ,,geschichtlichen`` Gründen gewachsen: Diese Klasse ist während ihrer Nutzung
entstanden und daher haben sich einige im Feuer des Gefechtes getroffenen
Erstentscheidungen als ungünstig erwiesen. Aus Kompatibilitätsgründen habe
ich allerdings einige dieser Entscheidungen ,,retten`` müssen. Ich werde
hier und dort beschreiben, was die ,,bessere`` Variante ist, aber auch die andere(n)
der Vollständigkeit halber angeben.
\end{description}

\section{Planung einer Unterrichtsstunde}
\subsection{Die Umgebung \{schedule\}}
Eine Tabelle mit der Stundenplanung kann mit Hilfe der Umgebung  \env{schedule} erstellt werden.
Ein typisches Beispiel könnte sein
\begin{verbatim}
    \begin{schedule}
    \time{45}
    \goal{Einstieg}
    \content{Irgendwas}
    \methode{UG}
    \material{DokuKamera}
    \newblock%%%%%%%%%%%%%%%%%%%%%
    \time{45}
    \goal{Ü}
    \content{Irgendwas Anderes}
    \methode{EA/PA}
    \end{schedule}
\end{verbatim}
Die verschiedenen Unterrichtsphasen (also faktisch die Tabellenzeilen) werden durch \cs{newblock}
getrennt. Hier kommt ein etwas größeres Beispiel:

%%%%%%%%%%%%%%%%%%%%%%%%%%%%%%%%%%%%%%%%%%%%%%%%%%%%%%%%%%%%%%%%%%%%%%%%%%%%%%%%%%%%%%%%%%
%%%%%%%%%%%%%%%%%%%%%%%%%%%%%%%%%%%%%%%%%%%%%%%%%%%%%%%%%%%%%%%%%%%%%%%%%%%%%%%%%%%%%%%%%%

\begin{schedule}[2]
\time{10}
\method{UG}
\goal{Einstieg}
\material{Folien}
\content{%
   Einstieg: Bilder zeigen und Diskussion\\
   Leitfrage: \textit{Wie entsteht Schatten?}\\
   \textbf{Frage}: Was brauchen wir, um mit Licht und Schatten zu experimentieren? \textrightarrow\
   Lichtquelle(n), undurchsichtigen Objekt, Schirm.
}%
\newblock%%%%%%%%%%%%%%%%%%%%%%%%%%%%%%%%%%%%%%%%%%%%%%%%%%%%%%%%%%%%%%%%%%%%%%%%%%%%%%%%%
\content{Die Reihenfolge der Befehle \cs{time}, \cs{goal}, \cs{material}, \cs{method} und
   \cs{content} ist irrelevant. Man kann selbstverständlich auch welche auslassen. (Außer
   \cs{time}, denke ich. Ich habe es ehrlich gesagt nie ausprobiert, weil es sowieso
   sinnlos ist \texttt{;-)})
   Das Makro \cs{newblock} trennt die verschiedenen Unterrichtsphasen.}
\time{5}
\material{Foo}
\method{Baz}
\newblock%%%%%%%%%%%%%%%%%%%%%%%%%%%%%%%%%%%%%%%%%%%%%%%%%%%%%%%%%%%%%%%%%%%%%%%%%%%%%%%%%
\time{10}
\content{%
   Im Argumenten von \cs{content} kann man folgende Sachen hinkriegen: Zum Beispiel einen
   \begin{TA}
   Tafelanschrieb. Der dicke Strich am Rand macht deutlicher, was zum TA gehört und was nicht.
   \end{TA}
   Die Umgebung \env{TA} nimmt auch einen optionalen Argumenten: \cs{begin\{TA\}[Titel]...} ergibt
   \begin{TA}[Titel]
   Ein Tafelanschrieb mit Titel. Die Schriftart vom Titel ist gespeichert im Makro \cs{TAtitlefont}
   (default\cs{scshape}\cs{itshape}).
   \end{TA}
   \V für Versuchsbeschreibung und \B für Beobachtung.\\
   Aus historischen Gründen ist es möglich, den Tafelanschrieb als \emph{Makro} mit
   einem optionalen und einem obligatorischen Argumenten anzugeben, d.h.\
   \cs{TA[Überschrift]\{Irgendein Text...\}} erzeugt
   \TA[Überschrift]{Irgendein Text, der nur da ist, um etwas Platz zu nehmen, um zu zeigen, wie sich das
   Ganze verhält, aber ohne einen besonderen tiefen Sinn. Nur Platzhalter halt.}
   Das ist \emph{sehr schlechter} \TeX-Stil aber es funktioniert\ldots
}
\material{}
\method{}
\newblock%%%%%%%%%%%%%%%%%%%%%%%%%%%%%%%%%%%%%%%%%%%%%%%%%%%%%%%%%%%%%%%%%%%%%%%%%%%%%%%%%
\time{35}
\method{EA/PA}
\content{%
   Die Tabelle für die Planung ist eigentlich ein \env{longtable}, d.h.\ sie kann über
   mehrere Seiten gehen. Ich fülle hier nur Zeug ein, um Platz zu nehmen, um zu zeigen,
   wie es auf der folgenden Seite aussieht. (Der Tabellenkopf wird wiederholt.)

   Es gibt ein Makro \cs{point}, der eine Art ,,poor-person-list`` einführt:
   \point Foo
   \point Bar
   \point Langer Text, um zu zeigen, dass der Text eingerückt wird, wenn er lang ist.\par
   Damit die Einrückung weg ist, muss man den Abschnitt explizit mit \cs{par} unterbrechen.
   Mit einem optionalen Parameter hat man etwas ähnlich wie \env{description}
   \point[Irgendwas] Lorem ipsum dolor sit et amet irgend ein Text, der keine besondere Bedeutung hat.\par
   (Das war die erste Idee hinter dem Makro für den Tafelanschrieb.)
}
\newblock%%%%%%%%%%%%%%%%%%%%%%%%%%%%%%%%%%%%%%%%%%%%%%%%%%%%%%%%%%%%%%%%%%%%%%%%%%%%%%%%%
\time{15}
\goal{Wdh}
\content{Dem geübten Auge fällt auf, dass das Makro \cs{time} eigentlich eine \TeX-Primitive
   ist. Diese wird in der Tabelle umdefiniert, aber innerhalb von \cs{contents\{...\}} wieder
   hergestellt, so dass \cs{the}\cs{time} ergibt \the\time.
   \begin{TA*}
   Es gibt auch eine Sternform \env{TA*} der Tafelanschrieb-Umgebung, die das fette
   ,,\textbf{TA}`` nicht schreibt und keinen optionalen Titel akzeptiert, sondern nur den
   dicken grauen Strich am Rand zeichnet.
   \end{TA*}
}
\material{Foo, bar baz, bla, meh}
\newblock%%%%%%%%%%%%%%%%%%%%%%%%%%%%%%%%%%%%%%%%%%%%%%%%%%%%%%%%%%%%%%%%%%%%%%%%%%%%%%%%%
\time{15}
\goal{Test}
\content{%
   Wenn sich insgesamt keine 90 Minuten ergeben (wie in diesem Fall), gibt es eine
   Warnung am Ende der Tabelle.}
\newblock%%%%%%%%%%%%%%%%%%%%%%%%%%%%%%%%%%%%%%%%%%%%%%%%%%%%%%%%%%%%%%%%%%%%%%%%%%%%%%%%%
\time{2}
\content{%
   Vorsicht: \env{schedule} ist letztendlich ein \env{longtable}: zwischen den verschiedenen
   Angaben \cs{material}, \cs{time} usw.~sollte \emph{keine leere Zeile} stehen. Diese wird
   sonst als neuer Abschnitt interpretiert und wird zu merkwürdigen Ergebnissen führen.
   \point[Nachtrag] Das müsste mit \opt{v0.3a} behoben worden sein, aber es ist trotzdem
   keine schlechte Idee, leere Zeilen zu vermeiden.
}
\end{schedule}

%%%%%%%%%%%%%%%%%%%%%%%%%%%%%%%%%%%%%%%%%%%%%%%%%%%%%%%%%%%%%%%%%%%%%%%%%%%%%%%%%%%%%%%%%%
%%%%%%%%%%%%%%%%%%%%%%%%%%%%%%%%%%%%%%%%%%%%%%%%%%%%%%%%%%%%%%%%%%%%%%%%%%%%%%%%%%%%%%%%%%

Zu Beginn der \env{schedule} Umgebung wird erst eine neue Seite gestartet, der Satzspiegel
etwas vergrößert, und eine Überschrift in der Form \emph{Klasse --- Titel --- Untertitel}
gedruckt. Diese Inhalte wurden in der Präambel folgendermaßen deklariert:
\begin{verbatim}
   \lesson{Thema}[Möglicher Untertitel]
   \class{8g}
\end{verbatim}
Der Untertitel ist natürlich optional und kann weggelassen werden. Hätte man die Befehle
in der Präambel nicht angegeben, so hätte man als Überschrift wörtlich
\begin{center}
Klasse 0 --- \cs{lesson\{Titel\}[Untertitel]}
\end{center}
bekommen (als kleine Erinnerung, wie man es verwenden soll).

Die Schulklasse ist voreingestellt auf die ,,0x``.
Die mit \cs{class} definierte Klasse wird gespeichert und kann mit dem Befehl \cs{printclass\optstar}
wiedergegeben werden. Die Sternform gibt nur die Klassenstufe, vergleiche
,,\printclass`` und ,,\printclass*``. Das Makro \cs{classname} speichert den ,,Namen`` der Klasse:
voreingestellt ist natürlich ,,\classname``.

Standardmäßig werden Blöcke von 90 Minuten angenommen. Die Dauer eines Unterrichtblocks
kann mittels \cs{SetDuration} geändert werden,~d.h.
\begin{verbatim}
    \SetDuration{45}
\end{verbatim}
legt grundsätzlich die Dauer eines Blocks auf 45~Minuten fest.

Im obigen Beispiel wurde die Umgebung mit dem optionalen Argumenten \opt{2} gestartet.
Das optionale Argument kann verschiedene Formen annehmen:
\begin{itemize}
\item Es kann eine Zahl sein, welche den Block/die Blöcke identifiziert,
in der/denen die Stunde stattfindet: \opt{1} für den ersten Block, \opt{2} für den zweiten usw., aber auch
\opt{12} für 1.\ und 2., \opt{134} für 1., 3.\ und 4., und alle möglichen Kombinationen.\footnote{%
   \opt{312} drückt erst den dritten, dann den ersten und zum Schluss den zweiten Block.
   Mit \opt{111} kriegt man dreimal den ersten Block. Man muss selbst etwas mitdenken\ldots
}
\item Alternativ kann man explizit eine Uhrzeit angeben
\bcode
   \\begin\{schedule\}[8:15]
\ecode
und dann beginnt die Planung zur gegebenen Uhrzeit.
\item Man kann auch eine Key-Value-Syntax verwenden. Die uninteressanten Beispiele sind\\[1ex]
\null\kern4em\cs{begin}\marg{schedule}\oarg{start=8:15} ist dasselbe wie \cs{begin}\marg{schedule}\oarg{8:15}\\[1ex]
und\\[1ex]
\null\kern4em\cs{begin}\marg{schedule}\oarg{block=13} ist dasselbe wie \cs{begin\{schedule\}[13]}\\[1ex]
Es gibt aber auch \emph{andere} Optionen, nämlich \opt{duration=...} und \opt{title=true/false}.
Man kann also angeben
\bcode
   \\begin\{schedule\}[start=8:00,duration=45]
\ecode
wenn man eine Einzelstunde will, die um 8 Uhr startet. Mit der Angabe \opt{title=false}
wird die Kopfzeile nicht gedruckt. (Default ist \opt{true}.) Natürlich ist es sinnlos,
sowohl \opt{block=...} als auch \opt{start=...} anzugeben: das letzte gewinnt.
\end{itemize}
Der Defaultwert des optionalen Argumenten ist \opt{1234}, d.h.~wenn gar kein
optionales Argument angegeben wird, so bekommt man den Stundenverlauf viermal
gedruckt (einmal pro Block).

\subsection{Neue Stile für die Stundenplanung definieren \texorpdfstring{\colorbox{green!30!white}}{}{(NEU! v1.0.0)}}
Wie im obigen Beispiel gezeigt, stehen innerhalb\footnote{%
   Und zwar \emph{nur} innerhalb. Außerhalb ist \cs{time} die bekannte \TeX\
   Primitive und alle anderen Makros sind nicht definiert.
}
einer Umgebung \env{schedule} die Befehle \cs{time}, \cs{goal},
\cs{material}, \cs{method} und \cs{content} zur Verfugung. Natürlich will
jede/r Fachleiter/in am Seminar etwas anderes haben, und so stehen
Anpassungsmöglichkeiten zur Verfügung. Jede \env{schedule} wird in
einem gegebenen \emph{Stil} gesetzt. Ein Stil wird durch
\bcode
   \\NewScheduleStyle\{\param{Name}\}[\param{relative Breite der Zeit-Spalte}]\{\param{Spaltendefinitionen}\}
\ecode
definiert. Im ersten Argumenten von \cs{NewScheduleStyle} steht der Name
des Stils; das zweite, optionale Argument beschreibe ich später. Im weiteren
obligatorischen Argumenten muss eine Reihe von verschiedenen Deklarationen der Form
\bcode
   \\DeclareColumn[\param{Extra-Code}]\{\param{Marko}\}\{\param{Überschrift}\}\{\param{relative Breite}\}
\ecode
sein; jede davon legt fest
\begin{compactenum}[(1)]
\item das Makro, das den Inhalt der Spalte setzt,
\item die entsprechende Spaltenüberschrift,
\item die \emph{relative} Breite der Spalte.
\end{compactenum}
Ein Beispiel: Der \opt{default}-Stil ist in etwa folgendermaßen definiert:
\begin{verbatim}
  \NewScheduleStyle{default}[5]{%
     \DeclareColumn{\goal}{Ziel}{6}%
     \DeclareColumn[\def\\{\newline}\let\time\TeXtime]{\content}{Inhalt}{35}%
     \DeclareColumn{\method}{Methode}{6}%
     \DeclareColumn{\material}{Material}{8}%
  }
\end{verbatim}
Es wird stets angenommen, dass die erste Tabellenspalte die Zeit beinhaltet
und standardmäßig die relative Breite~1 hat: diese ,,Referenzbreite`` kann
mit dem optionalen Argumenten geändert werden, hier im Beispiel~5.\footnote{%
   Wo kommt die 5 her? Historisch. Wenn ich jetzt den \opt{default}-Stil neu definieren
   würde, würde ich es vielleicht etwas anders machen, aber ich möchte nicht, dass
   plötzlich alle meine alten Dateien anders aussehen.}

Was passiert dann hier genau? Mit der obigen Definition hat die Tabelle in dem
\opt{default}-Still eine ,,Zeit``-Spalte mit Breite~5, eine ,,Ziel``-Spalte mit
Breite~6, eine ,,Inhalt``-Spalte mit Breite~35, eine ,,Methode``-Spalte mit
Breite~6 und eine ,,Material``-Spalte mit Breite~8.
Zu Beginn der \env{schedule} wird einfach alles addiert: $5+6+35+6+8=60$.
Die Klasse rechnet dann die entsprechende Breite jeder Spalte unter der Voraussetzung,
die Tabelle sei so breit wie der Satzspiegel (und berücksichtigt natürlich
die Tabellenlinien).

Das optionale Argument zu \cs{DeclareColumn} hat die folgende Funktion: Das
wie oben definierte Makro \cs{content} speichert zuerst seinen Inhalt in
einem Makro \cs{drc@content}. Dieses wird dann in der Tabelle an der geeigneten
Stelle platziert. Da aber in der Tabelle das Makro \texttt{\string\\} die neue
Tabellenzeile startet, kann man \texttt{\string\\} im Argumenten von \cs{content}
nicht verwenden, was natürlich blöd ist. Dazu ist das optionale Argument
zu \cs{DeclareColumn} da: es ist extra Code, das zu Beginn der Zelle kopiert wird
\bcode
   ... \& \param{extra Code}\\drc@content \& \\drc@notes ...
\ecode
Das heißt, in der ,,Inhalt``-Spalte kann man \texttt{\string\\} verwenden,
ohne dass eine neue Tabellenzeile gestartet wird (mit dem Chaos, das dabei
entstehen würde), und das Makro \cs{time} bekommt wieder seine ursprüngliche
Definition.

Man kann somit z.B.~einen anderen Stil, sagen wir \opt{simple}, definieren:
\begin{verbatim}
   \NewScheduleStyle{simple}{%
      \DeclareColumn[\def\\{\newline}\let\time\TeXtime]{\content}{Inhalt}{8}%
      \DeclareColumn{\notes}{Anmerkungen}{3}%
   }
\end{verbatim}
Wie kann man diesen Stil verwenden? Dazu gibt es zwei Möglichkeiten. Man kann
entscheiden, dass dieser Stil grundsätzlich angewandt werden soll, indem man in der
Präambel
\begin{verbatim}
    \UseScheduleStyle{simple}
\end{verbatim}
schreibt. Alternativ kann man im optionalen Argumenten der Umgebung \env{schedule}
\opt{style=\param{Name}} angeben. So ergibt zum Beispiel
\begin{verbatim}
    \begin{schedule*}[style=simple,start=8:00,title=false]
    \time{45}
    \content{Irgendwas}
    \notes{Upps}
    \newblock
    \time{45}
    \content{Irgendwas anderes}
    \notes{Mir fällt nichts ein}
    \end{schedule*}
\end{verbatim}
das folgende Ergebnis:
\NewScheduleStyle{simple}{%
   \DeclareColumn[\def\\{\newline}\let\time\TeXtime]%
       {\content}{Inhalt}{8}%
   \DeclareColumn{\notes}{Anmerkungen}{3}%
}
\begin{schedule*}[style=simple,start=8:00,title=false]
\time{45}
\content{Irgendwas}
\notes{Upps}
\newblock
\time{45}
\content{Irgendwas anderes}
\notes{Mir fällt nichts ein}
\end{schedule*}

Die hier verwendete Sternform \env{schedule*} startet keine neue Seite\footnote{%
   Na ja, manchmal schon. \env{schedule} ist letztendlich ein \env{longtable}
   und es kann trotzdem eigenmächtig entscheiden, eine neue Seite zu starten.}
 und ändert den
Satzspiegel nicht, verhält sich sonst in allem wie \env{schedule}.\footnote{%
   Beim \opt{default}-Stil ist aber die ,,Methode``-Spalte zu eng, und man bekommt
   entsprechend Warnungen. Das ist wieder ein Relikt aus der Entstehungsgeschichte.}

Natürlich kann man auch die Standardeinstellung ändern. Neben \cs{NewScheduleStyle}
existiert \cs{RenewScheduleStyle}, und man könnte mit
\begin{verbatim}
    \RenewScheduleStyle{default}{%
       \DeclareColumn[\def\\{\newline}\let\time\TeXtime]%
           {\content}{Inhalt}{8}%
       \DeclareColumn{\notes}{Anmerkungen}{3}%
    }
\end{verbatim}
den Defaultstil umdefinieren. Es wäre aber auch denkbar, dass man den Defaultstil
umdefinieren aber nicht unbedingt verlieren möchte: Man kann auch Stile kopieren.
Mit
\bcode
   \\CopyScheduleStyle\{\param{neuer Stil}\}\{\param{alter Stil}\}
\ecode
macht man eine Kopie eines existierenden Stils, der dann geändert werden kann.

\subsection{Uhrzeiten}
Die Uhrzeiten der Blöcke sind zuerst festgelegt auf 07:50, 09:40, 11:25 und~14:00
(die Uhrzeiten meiner Schule halt\ldots), und die Dauer eines Unterrichtblocks ist
wie schon gesagt auf 90~Minuten initialisiert. Wie bereits beschrieben, können die Startzeit
und die Unterrichtsdauer einer einzelnen Umgebung \env{schedule} mit den Optionen \opt{start=...}
und \opt{duration=...} geändert werden. Die Startzeiten der Blöcke können im Allgemeinen
mit Hilfe von \cs{SetBlockStart} festgelegt werden. Vordefiniert sind
\begin{verbatim}
    \SetBlockStart{1}{07:50}
    \SetBlockStart{2}{09:40}
    \SetBlockStart{3}{11:25}
    \SetBlockStart{4}{14:00}
    \SetBlockStart{5}{15:40}
\end{verbatim}
und die kann man natürlich nach Belieben umdefinieren.

\section{Individuelle Anpassungen speichern}\label{sec:anpassungen}
Es ist mir klar, dass jede Schule andere Uhrzeiten hat; und es ist mir auch klar,
dass es lästig wäre, eigene \cs{SetDuration} und \cs{SetBlockStart} in jede
Datei zu schreiben. Natürlich kann man alle Einstellungen in eine Datei \opt{meinemacros.tex}
speichern und dann \cs{input\{meinemacros\}} verwenden
(oder--eleganter--in einem Paket \opt{meineschule.sty} und dann \cs{usepackage\{meineschule\}}),
aber die Klasse bietet zwei
Möglichkeiten an, eigene Anpassungen zu definieren.

\subsection{Hauptkonfigurationsdatei}

Zuerst überprüft die Klasse immer, ob eine ,,Hauptkonfigurationsdatei`` namens
\texttt{drcschool.cfg} existiert, und lädt diese gegebenenfalls.
Nehmen wir daher an, Frau~Maier möchte ihre Einstellungen ein für alle Mal speichern.
Insbesondere hat sie in ihrer Schule Blöcke von 55~Minuten mit einer Pause zwischen
dritten und vierten, sowie zwischen fünften und sechsten. Außerdem möchte sie
eher den Stil \opt{simple} für die Unterrichtsplanung verwenden aber den
\opt{default}-Stil nicht überschreiben, denn er kommt bei Unterrichtsbesuchen
ihres Wasauchimmer-Fachleiters gut an. Darüber hinaus möchte Frau~Meier
Times New Roman als Schriftart verwenden.\footnote{Bleargh\ldots}
Um das alles zu machen, schreibt sie die Datei \texttt{drcschool.cfg}
\begin{verbatim}
   \ProvidesFile{drcschool.cfg}
   \SetLogo{example-image-a}
   \SetDuration{55}
   \SetBlockStart{1}{8:00}
   \SetBlockStart{2}{8:55}
   \SetBlockStart{3}{9:50}
   \SetBlockStart{4}{11:00}
   \SetBlockStart{5}{11:55}
   \SetBlockStart{6}{14:00}
   \SetBlockStart{7}{14:55}
   \SetBlockStart{8}{15:50}
   \NewScheduleStyle{simple}{%
      \DeclareColumn[\def\\{\newline}\let\time\TeXtime]%
          {\content}{Inhalt}{8}%
      \DeclareColumn{\notes}{Anmerkungen}{3}%
   }
   \UseScheduleStyle{simple}
   \NoFonts
   \RequirePackage{newtxtext,newtxmath}
\end{verbatim}
und platziert sie dort, wo \TeX\ sie finden kann
(am Besten da, wo sich auch \texttt{drcschool.cls} befindet).
Die erste Zeile (\cs{ProvidesFile}) ist nicht notwendig aber wärmstens empfohlen.
Die Bedeutung der zweiten Zeile (\cs{SetLogo}) wird später bei der Herstellung
von Klassenarbeiten oder Tests erklärt. Und da Frau~Maier eine Times-Schrift
verwenden will, schreibt sie \cs{NoFonts} in der Konfigurationsdatei, bevor sie
die Pakete \pkg{newtxtext} und \pkg{newtxmath} lädt. Natürlich kann Frau~Mayer
in dieser Datei auch alle ihre Lieblingspakete und persönlichen Definitionen
schreiben, und muss somit eine ewig lange Präambel nicht in jede ihrer Dateien
kopieren.

\subsection{Schulkonfigurationsdatei(en)}
Hauptsächlich aus Kompatibilitätsgründen existiert eine weitere Möglichkeit,
persönliche Konfigurationen zu Laden, nämlich mit Hilfe einer
Schulkonfigurationsdatei. Sagen wir, Frau~Meyer unterrichtet in zwei verschiedenen
Schulen, die verschiedene Uhrzeiten haben. Dann kann sie für jede Schule eine
Datei mit Endung ~\opt{.sco} (steht für \emph{school class option}) schreiben,
die auch dort platziert wird, wo \TeX\ sie finden kann. Sie sieht im Grunde
wie die Hauptkonfigurationsdatei aus
\begin{verbatim}
    \ProvidesFile{MeineSchuleA.sco}
    % alles, was man will
\end{verbatim}
muss aber in der Präambel mit
\begin{verbatim}
   \LoadSchoolOptionFile{MeineSchuleA}
\end{verbatim}
explizit geladen werden (\emph{ohne} \opt{.sco} Endung). Ähnliches
kann Frau Maier\footnote{Ist es jemandem aufgefallen, dass ich alle möglichen Kombinationen (ai/ai/ay/ey) verwendet habe? \texttt{:-P}}
mit ihrer anderen Schule machen.

In so einem Fall bietet sich auch eine Kombination aus beiden Methoden an:
In der Hauptkonfigurationsdatei \texttt{drcschool.cfg} wird der Code geschrieben,
der für
alle Schulen gelten soll, und in die beiden Schulkonfigurationsdateien
kommen die schulbezogenen
Einstellungen.
Die Hauptkonfigurationsdatei \texttt{drcschool.cfg}
wird am Ende der Klasse geladen, während die \opt{sco}-Dateien
in der Präambel geladen werden (genau da, wo \cs{LoadSchoolOptionFile} verwendet wird):
Eine Schulkonfigurationsdatei kann daher Befehle der Hauptkonfigurationsdatei
überschreiben.

Vorsicht: Es kann pro \TeX-Datei \emph{nur eine} \opt{sco}-Datei geladen werden!

%%%%%%%%%%%%%%%%%%%%%%%%%%%%%%%%%%%%%%%%%%%%%%%%%%%%%%%%%%%%%%%%%%%%%%%%%%%%%%%%%%%%%%%%%%
%%%%%%%%%%%%%%%%%%%%%%%%%%%%%%%%%%%%%%%%%%%%%%%%%%%%%%%%%%%%%%%%%%%%%%%%%%%%%%%%%%%%%%%%%%

\section{Arbeitsblätter, Klassenarbeiten \& Co.}
Es folgen nur einige Beispiele für die verschiedenen Arten von Arbeitsblättern, die mit
der Klasse erstellt werden können:
\begin{itemize}
\item Zuerst kommen mehrere Beispiele der Umgebung \env{worksheet}, die ein ,,normales``
      Arbeitsblatt erstellt. Ein Arbeitsblatt wird zweimal gedruckt: einmal mit und
      einmal ohne Lösung. Die folgenden Beispiele hätten natürlich auch in einer
      einzelnen \env{worksheet} Umgebung gesetzt werden können; sie aber in mehreren
      Beispielen aufzuteilen, hat den Vorteil, dass man die Variante mit Lösung nicht
      erst zehn Seiten später sieht.
\item Es kommen dann einige Beispiele von Umgebungen der Form \env{print\param{N}},
      (wobei \param{N} ist die Zahl~2, 3, oder~4), die ihren Inhalt mehrmals auf einer
      Seite drücken.
\item Es folgen dann Beispiele der Umgebung \env{cluecards}, mit der man z.B.~Lösungskärtchen drucken kann.
\item Zum Schluss kommt ein Beispiel für die Umgebung \env{test} für Tests/Klassenarbeiten.
      Die ist weitgehend ähnlich zu einem Arbeitsblatt aber mit der Möglichkeiten,
      Punktzahlen anzugeben.
\end{itemize}

\begin{description}
\item[Anmerkung] Features, die \emph{nur} mit \TikZ\ funktionieren, werden mit
\onlytikz\ markiert.
\end{description}

\begin{worksheet}[Titel des Arbeitsblatts]
\begin{colorminipage}{red!30!white}{\linewidth}
\centerline{\vrule width0pt height2ex depth1ex\bfseries!! Caveat emptor !!}
Arbeitsblätter werden zweimal abgedruckt: einmal ohne Lösung(en), und
einmal mit. Dafür muss die \env{worksheet} Umgebung ihren gesamten Inhalt
sammeln, was \cs{catcode}s einfriert!
(Das ist ein ziemlich \TeX nisches Detail: wer das versteht, weiß damit
umzugehen; wer's nicht versteht, wird vermutlich keine Probleme haben.)
\end{colorminipage}
\par\medskip\noindent
Arbeitsblätter werden mit Hilfe der Umgebung \env{worksheet} hergestellt.
Der Titel des Arbeitsblatts kann als optionales Argument zu \cs{begin}\env{worksheet}
gegeben werden. Das optionale Argument kann allerdings auch eine Key-Value-Liste
sein. Mögliche Optionen sind:
\begin{compactdesc}
\item[\opt{title=\param{Titel}}] legt den Titel fest;
\item[\opt{date=\param{Ein-Aus-Wert}}] legt fest, ob die Überschrift ,,Datum`` in der Kopfzeile angezeigt wird;
\item[\opt{name=\param{Ein-Aus-Wert}}]legt fest, ob der Name in der Kopfzeile geschrieben werden soll;
\item[\opt{fontsize=\param{Schriftgröße}}] ändert die Schriftgröße im Arbeitsblatt;
\item[\opt{geometry=\param{Optionen}}] gibt die \param{Optionen} an das \pkg{geometry} Paket weiter.
\end{compactdesc}
(Ein \param{Ein-Aus-Wert} ist wie gewohnt \opt{true}/\opt{yes}/\opt{on} oder \opt{false}/\opt{no}/\opt{off}.)
Im Grunde ist also
\bcode
\\begin\{worksheet\}[Titel des Arbeitsblatts]
\ecode
in etwa dasselbe wie
\bcode
\\begin\{worksheet\}[%
   \begin{tabular}[t]{@{}l}title=Titel des Arbeitsblatts,\%\\
                           date=true,name=false]\end{tabular}
\ecode
Wenn man die Standardeinstellungen ändern will, dann stehen die Befehle
\cs{SetWorksheetOptions} und \cs{AddWorksheetOptions} zur Verfügung.
Wenn z.B.~alle Arbeitsblätter grundsätzlich den Platz für den Namen
haben sollen, so kann man
\bcode
\\SetWorksheetOptions\{name=true,date=true\}
\ecode
in der Präambel oder in einer Konfigurationsdatei schreiben.

\exercise[Titel]\label{ex:foo}
Eine Aufgabe wird mittels \cs{exercise} gestartet.
Ein Titel kann als optionales Argument gegeben werden: \cs{exercise[Titel]}.
Innerhalb einer Aufgabe können einzelne Teilaufgaben mit Hilfe der Umgebung
\env{questions} gestellt werden:
\begin{questions}
\question Bla bla bla bla
\begin{hint}
Hier ein kleiner Hinweis.
\end{hint}
\question Am Ende einer einzelnen Frage (oder einer Aufgabe, falls es keine Fragen gibt)
   kann man eine kurze Lösung mit Hilfe von \cs{minisolution} angeben.\minisolution{43}
\end{questions}
Die Umgebung kann unterbrochen werden; wenn sie nochmal startet, läuft der Zähler
weiter:
\begin{questions}
\question* Noch eine.
\question Und noch eine.
\end{questions}
Wie man hier sieht, gibt es auch eine Sternform \cs{question*}:
Links vom Buchstabe erscheint dann eine Markierung, die in dem Makro
\cs{starredquestionmark} gespeichert ist (default eben \opt{*}).
\begin{hint}[Aufpassen!]
Ein Hinweis beginnt mit \cs{hintname}~(default ,,Hinweis``) gefolgt vom \cs{hintsep}~(default ,,:``).
Man kann diese Makros umdefinieren oder man kann das optional Argument zur \env{hint}-Umgebung
verwenden.
\end{hint}

\begin{solution}
Lösungen werden innerhalb der \emph{Umgebung} \env{solution} geschrieben.
Falls es \env{questions} gab, kann man entsprechend Teilantworten mittels \cs{answer}
angeben.
\answer Lösung der Teilaufgabe Lösung der Teilaufgabe
\[
x=2
\]
Lösung der Teilaufgabe Lösung der Teilaufgabe Lösung der Teilaufgabe.
\answer Lösung der Teilaufgabe Lösung der Teilaufgabe Lösung der Teilaufgabe Lösung der Teilaufgabe Lösung der Teilaufgabe.
\answer Wurde eine Frage mit der Sternform \cs{question*} angegeben, so erscheint der
Stern automatisch bei der entsprechenden Antwort.

Am Ende der \env{solution} Umgebung wird überprüft, ob die Anzahl der \cs{answer}s
der Anzahl der \cs{question}s entspricht; falls nicht (wie in diesem Fall) wird eine Warnung herausgegeben.
\end{solution}

\exercise[Multiple-Choice-Quiz]
Es stehen die Symbole \cs{checkbox} (\checkbox) und \cs{radiobutton} (\radiobutton)
zur Verfügung. Beide haben eine Sternform \cs{checkbox*} (\checkbox*) und
\cs{radiobutton*} (\radiobutton*), die die richtige (bzw.~anzukreuzende) Antwort
in der Variante mit Lösung druckt.

Auf diese Symbolen bauen zwei Umgebungen für Multiple-Choice-Aufgaben. Eine
verwendet die runden Radiobuttons, die andere dagegen die eckigen Auswahlkasten.
Typischerweise werden die Radiobuttons verwendet, wenn \emph{nur eine} Lösung
korrekt ist, und die Auswahlkasten, wenn mehrere Antworten möglich sind. Um das
Ganze schön verwirrender zu machen, habe ich mich entschieden, für die beiden
Umgebungen die \emph{englischen} Namen zu verwenden. Was im Deutschen als
,,Single-Choice`` bezeichnet wird, ist auf Englisch ,,\emph{multiple choice}``;
und was auf Deutsch ,,Multiple-Choice`` ist, heißt auf Englisch
,,\emph{multiple response}``.

Entsprechend ihrer gemeinten Nutzung geben beide Umgebungen eine Warnung,
wenn eine Frage keine als richtig markierte Antwort hat. Die Umgebung \env{multchoice}
gibt darüber hinaus eine Warnung, wenn eine Frage mehr als eine richtige Antwort
hat. Beide Umgebungen verwenden immer denselben Zähler wie auch die Umgebungen
\env{questions} und \env{questions*}, und können daher zusammen in einer Aufgabe
verwendet werden. (Ob das sinnvoll ist, ist eine andere Geschichte\ldots)

Die Umgebung \env{multchoice} verwendet Radiobuttons
\begin{multchoice}
\question Fragen werden durch \cs{question} gestellt,
   \choice mögliche Antworten durch \cs{choice}.
   \choice Falsch.
   \choice* Die richtige Wahl wird im Quellcode als \cs{choice*} gegeben.
\question* Noch eine Frage. Auch hier besteht die Möglichkeit, Fragen mit \opt{*} zu markieren.
   \choice Falsch.
   \choice* Richtig.
   \choice Nutzt man auch hier \cs{choice*}, so kriegt man einen Fehler.
\end{multchoice}
während \env{multresponse} Auswahlkasten benutzt und mehrere Antworten zulässt:
\begin{multresponse}
\question Frage
   \choice* Richtig.
   \choice Falsch.
   \choice* Auch richtig.
\question Noch eine:
   \choice Falsch.
   \choice* Richtig.
   \choice Falsch.
\end{multresponse}

\exercise[Kommas als Dezimaltrenner]
Das Paket \pkg{icomma} wird automatisch geladen. Man kann also das Komma als Dezimaltrenner
im Mathe-Modus verwenden, ohne unangenehme Leerräume zu kriegen: $2,5\,\mathrm{cm}$.
Will man allerdings das Komma als Interpunktionszeichen haben, dann muss man im Quellcode
einen Leerraum lassen: Man vergleiche
\begin{tabular}[t]{l@{\space\textrightarrow\space}l}
\opt{\$(a,b)\$}  & $(a,b)$ \\
\opt{\$(a,\textvisiblespace b)\$} & $(a, b)$
\end{tabular}

\end{worksheet}

%%%%%%%%%%%%%%%%%%%%%%%%%%%%%%%%%%%%%%%%%%%%%%%%%%%%%%%%%%%%%%%%%%%%%%%%%%%%%%%%%%%%%%%%%%
%%%%%%%%%%%%%%%%%%%%%%%%%%%%%%%%%%%%%%%%%%%%%%%%%%%%%%%%%%%%%%%%%%%%%%%%%%%%%%%%%%%%%%%%%%

\begin{worksheet}[Horizontale Auflistungen und Wahr-Falsch-Tabellen]

In Mathe-Büchern werden üblicherweise Teilaufgaben horizontal aufgelistet. Ankreuzaufgaben
In Mathe-Büchern werden Teilaufgaben üblicherweise horizontal aufgelistet.
Ankreuzaufgaben mit relativ kurzen Texten könnten auch platzsparender auf
Spalten verteilt werden. Zu diesem Zwecke gibt es jeweils eine Sternform
\env{questions*}, \env{multresponse*} und \env{multchoice*}: diese erwarten
die Anzahl der Spalten als obligatorisches Argument.

\exercise[Fragen wie im Lambacher-Schweizer]
Zum Beispiel, mit \cs{begin\{questions*\}\{4\}} bekommt man
\begin{questions*}{4}
\question AAA
\question* BBB
\question CCC
\question* DDD
\question EEE
\question FFF
\question GGG
\end{questions*}
Auch diese Umgebung kann unterbrochen werden, und eine weitere Umgebung
zählt einfach weiter:
\begin{questions*}{3}
\question HHHHHH
\question IIIIII
\question JJJJJJ
\end{questions*}
Die Sternform \cs{question*} funktioniert selbstverständlich auch hier.

Sowohl \env{questions} als auch \env{questions*} nutzen denselben Zähler, so
dass beide in einer Aufgabe zusammen verwendet werden können:
\begin{questions}
\question Text einer ,,normalen`` Aufgabe, die in einer gewöhnlichen Liste gesetzt wird.
\question Text einer weiteren ,,normalen`` Aufgabe, die in einer gewöhnlichen Liste gesetzt wird.
\end{questions}

\exercise[Ankreuzaufgaben]
Dasselbe funktioniert mit \env{multchoice*}
\begin{multchoice*}{3}
\question Foo
   \choice* A
   \choice B
   \choice C
\question Foo
   \choice A
   \choice B
   \choice* C
\question Foo
   \choice A
   \choice* B
   \choice C
\end{multchoice*}
sowie mit \env{multresponse*}
\begin{multresponse*}{3}
\question Foo
   \choice* A
   \choice B
   \choice* C
\question Foo
   \choice A
   \choice B
   \choice* C
\question Foo
   \choice* A
   \choice* B
   \choice C
\question Foo
   \choice A
   \choice* B
   \choice* C
\question Foo
   \choice A
   \choice* B
   \choice C
\end{multresponse*}

\exercise[Tabellenabstände verfeinern]
Alle diese horizontalen Auf{}listungen sind letztendlich Tabellen, und entsprechend
kann zwischen den Zeilen \emph{kein Seitenumbruch} erfolgen.\footnote{%
   Ich habe auch nicht vor, in nächster Zeit die Möglichkeit eines
   Seitenumbruchs zuzulassen, denn ich finde es nicht wirklich sinnvoll.
   (Na gut, bei Ankreuzfragen vielleicht schon\ldots)
} Es kann allerdings auch so etwas passieren:
\begin{questions*}{4}
\question $\dfrac{48}{12}$
\question $\dfrac{48}{12}$
\question $\dfrac{48}{12}$
\question $\dfrac{48}{12}$
\question $\dfrac{48}{12}$
\question $\dfrac{48}{12}$
\question $\dfrac{48}{12}$
\end{questions*}
Suboptimal eben\ldots\ Man könnte das Ergebnis verbessern, wenn man \cs{extrarowheight}
ändern könnte, oder wenn man neue Zeilen mit \str\\\oarg{\param{Länge}} machen könnte,
aber all dies ist ,,versteckt`` hinter der Umgebung. Dafür gibt es eine Schnittstelle:
\bcode
\\begin\{questions*\}[\param{Key-Value-Liste}]\{\param{Anzahl der Spalten}\}
\ecode
mit folgenden Keys:
\begin{compactdesc}
\item[\opt{extrarowheight=\param{Länge}}] wird zu \cs{setlength}\marg{\cs{extrarowheight}}\marg{\param{Länge}}
\item[\opt{arraystretch=\param{Zahl}}] wird zu \cs{renewcommand*}\marg{\cs{arraystretch}}\marg{\param{Zahl}}
\item[\opt{tabcolsep=\param{Länge}}] wird zu \cs{setlength}\marg{\cs{tabcolsep}}\marg{\param{Länge}}
\item[\opt{extrarowdepth=\param{Länge}}] führt \str\\\oarg{\param{Länge}} bei jeder neuen Zeile ein
\item[\opt{lastdepth=\param{Länge}}] führt \str\\\oarg{\param{Länge}} nach der \emph{letzten} Zeile ein.
\end{compactdesc}
Die letzte Zeile muss leicht anders betrachtet werden. In der Tat, mit der Option \opt{extrarowheight}
wird auch dieser Parameter fixiert. Setzt man das obige Beispiel mit
\bcode
\\begin\{questions*\}[extrarowdepth=2ex,extrarowheight=1.5ex]\{4\}
\ecode
so kriegt man
\begin{questions*}[extrarowdepth=2ex,extrarowheight=1.5ex]{4}
\question $\dfrac{48}{12}$
\question $\dfrac{48}{12}$
\question $\dfrac{48}{12}$
\question $\dfrac{48}{12}$
\question $\dfrac{48}{12}$
\question $\dfrac{48}{12}$
\question $\dfrac{48}{12}$
\end{questions*}
was etwas besser aussieht. Es gibt natürlich keine allgemeine Regel, welche
Kombination aus Längen und Streckfaktor am Besten aussieht. Da muss man leider
etwas herumprobieren.

\exercise[Wahr-Falsch-Tabelle]
Eine wahr/falsch Tabelle wird mittels der Umgebung \env{TF} erzeugt: Dies ist im Grunde
eine \env{tabularx}, und jede Aussage muss mit \cs{true} oder \cs{false} beendet werden.
Neue Zeilen werden automatisch hinzugefügt (es sind also keine \cs{\textbackslash} nötig),
horizontale Linien (mit \cs{hline} oder \cs{midrule}) sind nach Geschmack natürlich möglich.
\begin{TF}
Aussage 1 \true
\midrule
Aussage 2, die sehr sehr lang ist, so dass der Text mehr als eine Zeile braucht, um gesetzt zu werden \false
\midrule
Aussage 3 \false
\midrule
Aussage 4 \true
\end{TF}
Natürlich kann man die ganzen \cs{midrule}s (oder \cs{hline}s, wenn man will) weglassen,
wenn man sie nicht will\ldots\ Die Gesamtbreite der Tabelle kann als optionaler Parameter
angegeben werden (default \texttt{\textbackslash linewidth}):
\begin{TF}[.35\linewidth]
Aussage A \true
Aussage B \false
\end{TF}

\end{worksheet}

%%%%%%%%%%%%%%%%%%%%%%%%%%%%%%%%%%%%%%%%%%%%%%%%%%%%%%%%%%%%%%%%%%%%%%%%%%%%%%%%%%%%%%%%%%
%%%%%%%%%%%%%%%%%%%%%%%%%%%%%%%%%%%%%%%%%%%%%%%%%%%%%%%%%%%%%%%%%%%%%%%%%%%%%%%%%%%%%%%%%%

\begin{worksheet}[Schwierigkeitssymbole und sonstige Teilaufgaben]

\easy\exercise
Eine einfache Aufgabe kriegt man mit \cs{easy}\cs{exercise}.

\medium\exercise[Bla]
Mittelschwere Aufgaben kommen aus \cs{medium}\cs{exercise[Bla]}. Natürlich kann man auch
einen Titel angeben.

\hard\exercise
Schwere Aufgaben können mit \cs{hard}\cs{exercise} erzeugt werden.
Die Schwierigkeitssymbole sind denen aus dem Lambacher-Schweizer
nachempfunden.

\harder\exercise[Noch härter]
Nicht im Lambacher-Schweizer aber trotzdem da: noch schwierigere Aufgaben
können mit Hilfe von \cs{harder}\cs{exercise} erstellt werden. Und ja, es
gibt auch (eher zum Spaß)\ldots

\deadly\exercise
\ldots\ auch \cs{deadly}\cs{exercise} \texttt{;-)}

\exercisesymbol{$\heartsuit\varheartsuit\heartsuit$}
\exercise[Putzig, gell?]
Mit dem Makro \cs{exercisesymbol\{...\}} kann man einen beliebigen Inhalt neben
der Überschrift. Das betrifft allerdings nur die unmittelbar darauf{}folgende Aufgabe.
\begin{solution}
Bitte beachten: Die Makros \cs{easy}, \ldots, \cs{deadly} dürfen \emph{nur}
vor dem Befehl \cs{exercise} verwendet werden. Der Code
\bcode
   \\hard\\SomeOtherMacro
\ecode
erzeugt einen Fehler.
Will man die Symbole irgendwo haben, so gibt es dafür die Befehle
\cs{easysymbol} (\easysymbol), \cs{mediumsymbol} (\mediumsymbol), \cs{hardsymbol} (\hardsymbol),
\cs{hardersymbol} (\hardersymbol), \cs{deadlysymbol} (\deadlysymbol).
Aus Kompatibilitätsgründen gibt es noch das Makro \cs{difficulty\{\param{Zahl}\}},
das eine Zahl von~0 bis~4 annimmt.
\end{solution}

\exercise[,,Die bereitgestellten Features sing gut, aber\ldots]
\ldots\ aber ich möchte eine Frage haben, die nicht in \env{questions} \& Co.\ auftaucht``.

Gut, das kann ich sogar verstehen. Das Makro \cs{question} funktioniert nur in
\env{questions} oder in \env{mult...} Umgebungen, was letztendlich Listen sind.
Die entsprechenden Sternformen sind Tabellen. Beides ist nicht 100\% flexibel,
das gebe ich zu. Versucht man, das Makro \cs{question} irgendwo zu verwenden,
so kriegt man einen Fehler.

Will man eine \cs{question} einfach so haben, so kann man \cs{lonelyquestion}
verwenden. Dies erhöht den Zähler und druckt den Buchstaben genau da, wo
man den Befehl verwendet: \lonelyquestion tu irgendwas\ldots

\begin{solution}
\answer Wozu? Wenn man zum Beispiel einen Text hat, der ein Bild umwickelt,
wie zum Beispiel mit \env{wrapfigure}. Da funktionieren Listen nicht so gut,
und \cs{lonelyquestion} schafft Abhilfe.
\end{solution}

\runinexercises
\exercise[Keine neue Zeile] Mit dem Macro \cs{runinexercises} werden Aufgabenüberschriften
als ,,run-in`` gesetzt, also keine neue Zeile wird danach gestartet. Als optionales
Argument kann man eine explizite Länge angeben: \cs{runinexercises}\oarg{\param{Länge}}
(default \texttt{0.8em}). Das Makro kann natürlich in der Präambel oder in einer
Konfigurationsdatei verwendet werden. ,,Normale`` (,,hanging``) Überschriften
bekommt man wieder mit \cs{hangexercises}\oarg{\param{Länge}} (default \texttt{.2ex plus.2ex minus.2ex}).

\end{worksheet}

%%%%%%%%%%%%%%%%%%%%%%%%%%%%%%%%%%%%%%%%%%%%%%%%%%%%%%%%%%%%%%%%%%%%%%%%%%%%%%%%%%%%%%%%%%
%%%%%%%%%%%%%%%%%%%%%%%%%%%%%%%%%%%%%%%%%%%%%%%%%%%%%%%%%%%%%%%%%%%%%%%%%%%%%%%%%%%%%%%%%%

\begin{worksheet}[Lückentexte und Zusatzaufgaben]

\exercise[Einfache Lückentexte]
Platz für Lückentexte wird mit dem Makro \cs{fillhere} eingeführt: \fillhere{ein Wort}.
Standardmäßig ist der Strich zweimal so lang wie der gesetzte Inhalt (eine Art
Handschriftkorrektur). Mit einem optionalen Parameter kann dieser ,,Streckfaktor``
geändert werden. Alternativ kann man als optionalen Parameter eine explizite Länge angeben,
und dann wird diese verwendet:\par\noindent
\begin{tabular}[t]{@{}l@{\space\textrightarrow\space}ll}
\cs{fillhere\{hallo\}} & \fillhere{hallo} \\
\cs{fillhere[2]\{hallo\}} & \fillhere[2]{hallo} & (dasselbe wie oben)\\
\cs{fillhere[2.5]\{hallo\}} & \fillhere[2.5]{hallo} & (Dezimalwerte sind auch möglich) \\
\cs{fillhere[2,5]\{hallo\}} & \fillhere[2,5]{hallo} & (Punkt oder Komma ist egal) \\
\cs{fillhere[3cm]\{hallo\}} & \fillhere[3cm]{hallo} & (explizite Länge geht auch)
\end{tabular}\par\noindent
Es gibt auch eine Sternform \cs{fillhere*\{...\}}: \fillhere*{bla bla}.\\
Der erzeugte Strich reicht bis zur Ende der aktuellen Zeile.
Es gibt absolut \emph{keine} Kontrolle darüber, dass der Text reinpasst. Es
könnte also so etwas passieren: \fillhere*{ein sehr langer Text, der so lange ist, dass er eigentlich aus der Zeile herausragt.}

\exercise[Die Umgebung \env{cloze}]
\begin{cloze}
Wie man in der vorausgehenden \fillhere{Aufgabe} sieht, kann
die \fillhere{Tiefe} der \fillhere{Unterstriche} möglicherweise
mit den Buchstaben der unterliegenden \fillhere{Zeile} kollidieren.
Außerdem ist es mit festen \fillhere{Lücken} sehr schwierig, einen
bündig ausgerichteten Text hinzukriegen.

Aus diesem Zweck steht die Umgebung \env{cloze} zur Verfügung. Im Grunde
ist es eine \env{flushleft} Umgebung mit einem etwas erhöhten \fillhere{Zeilenabstand}. Dieser wird
standardmäßig um einen Faktor 1.4 erhöht, was mit einem optionalen Parameter geändert werden
kann. Für einen \fillhere{größeren} Zeilenabstand kann man z.B.\ \cs{begin\{cloze\}[1.6]} verwenden.
\end{cloze}

\exercise[Lückentexte als graue Boxen]
Das Makro \cs{fillhere} gibt ihren Inhalt immer im Textmodus wieder. Für den
Mathe-Modus ist es eh nicht geeignet, da der Unterstrich in Formeln missverstanden
werden könnte. Aus diesem Grunde gibt es für Mathe-Ausdrücke (aber es funktioniert
natürlich auch im Text) das Makro \cs{fillbox}: Dies druckt einen grauen
Kasten, der standardmäßig auch zweimal so breit als die ,,natürliche``
Größe des Textes ist; genau wie bei \cs{fillhere} kann der Streckfaktor mit Hilfe
des optionalen Argumenten geändert werden bzw.~explizit als Länge deklariert werden:
siehe \fillbox{hier}, oder siehe \fillbox[4]{hier}. In Gegensatz zu \cs{fillhere}
skaliert allerdings \cs{fillbox} in Mathe-Modus
\[
3^{\fillbox{2}} = 9, \qquad \fillbox{49}^{1/2}=7
\]
und ist somit für mathematische Ausdrücke besser geeignet.

\exercisesymbol{\onlytikz}
\exercise[Karierte Felder]
Noch etwas Nützliches: Der Befehl \cs{grid(\param{x-dimen},\param{y-dimen})\{...\}} erzeugt ein Kastenfeld, dessen Inhalt nur in
der Lösung angezeigt wird. Die Baseline der ersten Zeile ist dieselbe des umgebenden
Textes. Zum Beispiel erzeugt \cs{grid(2,1)\{Text hier drin.\}} dies: \grid(2,1){Text hier drin.}
ein Kastenfeld 2\,cm breit und 1\,cm hoch. Man kann auch explizite Größen angeben,
aber Vorsicht! Das Gitter hat 5mm-Schritte, und wenn man kein Vielfaches eines
halben Zentimeters angibt, passiert dies: \grid(27mm,.5){Hello}.

Man kann \cs{grid} auch ohne explizite Maße verwenden: das Kastenfeld ist dann
fast so breit wie die Textbreite (trunkiert auf halbe Zentimeter) und 3,5\,cm hoch:

\grid{%
Hier kann eine längere Lösung stehen. Gleichungen gehen auch:
\[
x=r\cos\phi
\]
VORSICHT! Es gibt (noch) keine Kontrolle, dass der Text im Gitter hereinpasst!

Das \cs{grid} beginnt immer mit einem \cs{noindent}, so wird es nie eine Einrückung geben.
}

\noindent
Will man ein Gitter mit der maximalen Breite aber einer anderen Höhe, kann man die
Breite als~\opt{*} angeben und die Höhe explizit: \cs{grid(*,0.5)\{...\}} erzeugt

\grid(*,.5){Dies wurde durch \cs{grid(*,.5)\{Dies wurde...\}} erzeugt.}

\noindent
Alternativ kann das erste Argument \opt{+} sein: dann wird das Gitter nicht trunkiert:

\grid(+,.5){Freilich könnte man dasselbe mit \cs{grid(\string\linewidth,0.5)\{...\}} erreichen.}%

\noindent
Wie bereits erwähnt, wird der Inhalt von \cs{grid} nur in der Variante mit Lösung gedrückt.
Manchmal will man allerdings doch etwas haben (z.B.~,,Merke``, oder ,,Beobachtung``, oder\ldots):
dafür kann man das optionale Argument verwenden, z.B.~\cs{grid[\string\bfseries\ Merke: ](*,1)\{Bla bla\}} ergibt

\grid[\bfseries Merke: ](*,1){Bla bla.}

\exercisesymbol{\onlytikz}
\exercise[Linien]
Ähnlich zu \cs{grid} gibt es \cs{lines}. Es funktioniert gleich, hat die gleiche Syntax,
und versucht sogar, den Zeilenabstand anzupassen (funktioniert aber nur für reinen Text):

\lines[\bfseries Schreibe hier etwas: ](*,2){Und was genau soll ich schreiben? Irgendein Text kommt hier,
so als Füller, sozusagen, ohne besonders tiefe Bedeutung.

Ach ja, hier gibt es keinen Unterschied zwischen \opt{+} und \opt{*} für die Breite,
denn für Linien ist die Trunkierung nicht relevant.}

\exercise*[Zusatzaufgabe]
Die Variante \cs{exercise*} (z.B.~für Zusatzaufgaben) markiert die Aufgabe mit einem
Symbol, das im Makro \cs{starredexercisemark} gespeichert ist. Das verwendete Symbol
kann auf übliche Weise mit \cs{renewcommand} umdefiniert werden, z.B. mit
\bcode
   \\renewcommand*\{\\starredexercisemark\}\{\\textsuperscript\{+\}\}
\ecode
bekommt man in der folgenden Aufgabe ein Plus anstatt von einem Sternchen.

\renewcommand{\starredexercisemark}{\textsuperscript{+}}
\medium\exercise*[Klar?]
Man kann natürlich \cs{hard} \& Co.\ auch verwenden, sowie einen Titel
angeben. Es wäre aber auch möglich, dass das Sternchen nicht eindeutig
ist und lieber ,,Zusatzaufgabe`` verwendet wird. Geht auch mit:
\bcode
   \\renewcommand*\{\\starredexercisemark\}\{\}\newline
   \\renewcommand*\{\\starredexercisename\}\{Zusatzaufgabe\}
\ecode
wird die nächste Zusatzaufgabe so aussehen:

\renewcommand*{\starredexercisemark}{}
\renewcommand*{\starredexercisename}{Zusatzaufgabe}
\exercise*[Ist es nun klar?]
Übrigens: Genau so wie es ein Makro \cs{starredexercisename} gibt, so gibt es auch
\cs{exercisename}. Beide sind zuerst auf ,,Aufgabe`` initialisiert. Ähnlich gibt
es ein Makro \cs{solutionname}, das auf ,,Lösung`` initialisiert wird.

\end{worksheet}

%%%%%%%%%%%%%%%%%%%%%%%%%%%%%%%%%%%%%%%%%%%%%%%%%%%%%%%%%%%%%%%%%%%%%%%%%%%%%%%%%%%%%%%%%%
%%%%%%%%%%%%%%%%%%%%%%%%%%%%%%%%%%%%%%%%%%%%%%%%%%%%%%%%%%%%%%%%%%%%%%%%%%%%%%%%%%%%%%%%%%

\begin{worksheet}[Verschiedenes]

\exercisesymbol{\onlytikz}
\exercise[Zuordnug-Quizzen]
Eine weitere nützliche (?) Umgebung ist \env{matching} für Zuordnung-Quizzen. Innerhalb der Umgebung
werden mehrere Befehle \cs{match} mit zwei Argumenten angegeben, als in
\bcode
   \\begin\{matching\}[\param{key=val}]\newline
   \\match\{aaa\}\{AAA\}\newline
   \\match\{bbb\}\{BBB\}\newline
   \\match\{ccc\}\{CCC\}\newline
   \\match\{ddd\}\{DDD\}\newline
   \\match\{eee\}\{EEE\}\newline
   \\end\{matching\}
\ecode
Diese Paare werden in zwei Spalten geordnet; die rechte Spalte
wird mit dem Fisher-Yates-Algorithmus zufällig angeordnet und
(in der Lösung) mit Pfeilen verbunden:
\begin{center}
\begin{matching}
\match{aaa}{AAA}
\match{bbb}{BBB}
\match{ccc}{CCC}
\match{ddd}{DDD}
\match{eee}{EEE}
\end{matching}
\end{center}
Die möglichen Optionen sind:
\begin{compactdesc}
\item[\opt{xsep=\param{Länge}}] setzt den Abstand zwischen den beiden Spalten (default 3\,cm);
\item[\opt{ysep=\param{Länge}}] setzt den Linienabstand (default \opt{1,3}\cs{baselineskip});
\item[\opt{bent}] verwendet gebogene statt gerade Linien;
\item[\opt{shuffle=\param{\opt{right} or \opt{left} or \opt{both}}}] legt fest, welche Spalte umsortiert wird
(\opt{shuffle} allein ist dasselbe wie \opt{shuffle=both});
\item[\opt{seed=\param{Zahl}}] initialisiert den Zufallszahlgenerator (default $\cs{time} \cdot \cs{year}$:
das ändert sich daher von Minute zu Minute).
\end{compactdesc}
Also mit
\bcode
   \\begin\{matching\}[shuffle,bent,seed=13974,xsep=2cm,ysep=.6cm]
\ecode
bekommt man für das Beispiel oben
\begin{center}
\begin{matching}[shuffle,bent,seed=13974,xsep=2cm,ysep=.6cm]
\match{aaa}{AAA}
\match{bbb}{BBB}
\match{ccc}{CCC}
\match{ddd}{DDD}
\match{eee}{EEE}
\end{matching}
\end{center}
Bemerkung: \env{matching} startet keine neue Zeile oder Abschnitt, sondern setzt
die Tabelle einfach da, wo der Code aufgerufen wird.

\exercise[Neue Tabellenspalten]
Die Klasse definiert einige besondere Tabellenspalten: ähnlich zu \opt{l},
\opt{c} und \opt{r} gibt es \opt{L}, \opt{C} und \opt{R}, die ihren Inhalt
direkt in (\opt{\string\displaystyle}) Mathe-Modus setzen, so dass man
nicht in jeder Zelle \opt{\$\string\displaystyle...\$} tippen muss.

Eine \opt{s}-Spalte ist eine \opt{c}-Spalte, die nur mit der Lösung angezeigt
wird (\opt{s} steht für \emph{solution}). Analog ist \opt{S} eine \opt{s}-Spalte
in Mathe-Modus. Diese sind bequem, damit man nicht \cs{solution} in jeder Zelle
einer Spalte tippen muss. Es ergibt sich allerdings ein Problem: eine auszufüllende
Tabelle hat meistens eine Kopfzeile, deren Inhalt immer sichtbar sein sollte.
Das Makro \cs{scolumnheader} ist dafür gedacht, eben im Header einer Tabelle
verwendet zu werden, damit der Inhalt immer gedruckt wird. Freilich kann man
das Makro etwas missbrauchen und in einer beliebigen Zelle (einer \opt{s}
oder \opt{S} Spalte) verwenden.

Ganz analog gibt es dann auch \opt{f}- und \opt{F}-Spalten, denen \cs{fillhere}
zu Grunde liegt. Diese brauchen einen obligatorischen Parameter für die Breite.

Am einfachsten ist ein explizites Beispiel:

\begin{tabular}[t]{lcrLCRsS}
\toprule
l   & c   & r   & L   & C       & R     & \scolumnheader Header hier & \scolumnheader Header hier \\
\midrule
ll  & cc  & rr  & L=L   & C=C   & R=R   & Lösung hier                & a^b                              \\
lll & ccc & rrr & LL=LL & CC=CC & RR=RR & Lösung hier                & a^b                              \\
\bottomrule
\end{tabular}

\exercise[Arbeitsblatt entweder mit oder ohne Lösung]
Es gibt auch eine Sternform \env{worksheet*}, die alle Key-Value-Optionen
annimmt, die für \env{worksheet} definiert sind, plus eine weitere: das Blatt
wird nur einmal gedruckt, entweder mit oder ohne Lösung, was mit der Option
\begin{compactdesc}
\item[\opt{solution=\param{Ein-Aus-Wert}}]
\end{compactdesc}
gesteuert wird. Standardmäßig wird \emph{keine} Lösung gezeigt.
Ich verwende das z.B.~wenn ich einfach einen längeren Text austeilen will.
Freilich, dafür bräuchte man keine besondere Umgebung und man könnte einfach in
der \opt{.tex}-Datei den Text aufschreiben. Genau so wie \env{worksheet} bietet
\env{worksheet*} allerdings die Möglichkeit an, Titel/Name/Datum/Schriftgröße
festzulegen.

\exercise[Plots]
Die Klasse lädt automatisch \pkg{pgfplots} (es sei denn, man verwendet die
Klassenoption \opt{nopgfplots}\ldots): die Umgebung \env{plot} ist
ein dünner Wrapper um \env{axis} mit einigen (für mich sinnvolle) Voreinstellungen
(siehe neben dem Plot).

\noindent
\begin{plot}[xmin=-5,xmax=5,ymin=-3.7,ymax=4.5]
\addplot[thick] {sin(deg(x))} ;
\IfSolutionT{\addplot[red,domain=-2.5:3]{.5*x};%
   \addplot[blue,domain=-2.5:3]{x};%
   \addplot[teal,domain=-3:2.5]{-1.5*x};%
   \node[circle,fill,label={-45:{$P_1$}},inner sep=2pt] at (axis cs:0,0) {};
   \node[circle,fill,label={-45:{$P_2$}},inner sep=2pt] at (axis cs:4,2) {};
}
\end{plot}\hfill
\parbox[b]{.3\linewidth}{%
   \footnotesize
   \textbf{Voreinstellungen:}\\
   \ttfamily
   compat=1.16,\%\\
   grid style=gray,\%\\
   axis line style=thick,\%\\
   no markers,\%\\
   x=1cm,\%\\
   y=1cm,\%\\
   xlabel style=\{right\},\%\\
   ylabel style=\{right\},\%\\
   xlabel=\{\$x\$\},\%\\
   ylabel=\{\$y\$\},\%\\
   grid=both,\%\\
   samples=50,\%\\
   axis lines=middle,\%\\
   xtick=\{-10,-9,...,10\},\%\\
   minor x tick num=\{1\},\%\\
   ytick=\{-10,-9,...,10\},\%\\
   minor y tick num=\{1\},\%\\
   major tick style=\\\null\qquad\{very thick,black\},\%\\
   minor tick style=\{draw=none\}
}

\noindent
Alle \pkg{tikz} und \pkg{pgfplots} Optionen können angegeben werden.

Wie man in dem Code für das Plot sieht, gibt es ein Makro \cs{IfSolutionT}, das
seinen Inhalt nur in der Variante mit Lösung zeigt. Das Makro kann überall verwendet
werden. \IfSolutionT{(Zum Beispiel hier.)\space\ignorespaces} Tatsächlich handelt
es sich um eins von vier verwandten Makros:
\begin{compactitem}
\item \cs{IfSolutionT\{arg\}} zeigt das Argument nur in der Variante mit Lösung,
\item \cs{IfSolutionF\{arg\}} zeigt das Argument nur in der Variante ohne Lösung,
\item \cs{IfSolutionTF\{arg1\}\{arg2\}} zeigt das Argument \opt{arg1} nur in der Variante mit Lösung,
und das Argument \opt{arg2} nur in der Variante ohne Lösung,
\item \cs{IfSolutionFT\{arg1\}\{arg2\}} zeigt das Argument \opt{arg1} nur in der Variante ohne Lösung,
und das Argument \opt{arg2} nur in der Variante mit Lösung.
\end{compactitem}

\exercise[Vierfeldertafeln]
Das Makro \cs{crosstable} setzt sein Argument in einer Vierfeldertafel:
\[
\crosstable{%
   1 & 2 & 3 \\
   4 & 5 & 9 \\
   5 & 7 & 12
}
\]
Der Inhalt der Vierfeldertafel wird erst gemessen, damit alle Zellen die gleiche
Größe haben (s.~nächstes Beispiel). Die zwei Ereignisse sind standardmäßig~$A$
und~$B$, aber man kann sie mit dem optionalen Argumenten ändern: das muss zwei
von einem Komma getrennten Bezeichnungen haben: Will man
also $R$ und~$M$, so muss man \cs{crosstable[R,M]} verwenden:
\[
\crosstable[R,M]{\dfrac23 & \dfrac16 & \IfSolutionT{\dfrac56} \\ zweite & Zeile & hier \\ hier & die & Summe}
\]
Natürlich funktionieren in der Tabelle \cs{IfSolutionT} \& Freunde.

\exercise[Schmutzige Tricks\ldots]
Wie in der Lösung von Aufgabe~\ref{ex:foo} gesagt, bekommt man eine Warnung,
wenn die Anzahl der \cs{question}s und die der \cs{answer}s nicht übereinstimmt.
Das kann passieren, wenn z.B.~in der gleichen Aufgabe eine Multiple-Choice-Aufgabe
und eine \env{questions\optstar} Umgebung gibt.
\begin{multchoice*}{3}
\question Frage A
\choice Antwort 1
\choice* Antwort 2
\choice Antwort 3
\question Frage B
\choice* Antwort 1
\choice Antwort 2
\choice Antwort 3
\question Frage C
\choice Antwort 1
\choice Antwort 2
\choice* Antwort 3
\end{multchoice*}
\begin{questions}
\question Und hier noch was.
\end{questions}
Nicht, dass das sinnvoll wäre, aber muss man mit der Warnung leben?
\begin{solution}
Es gibt mehrere Auswege. Man kann natürlich so was schreiben wie:
\bcode
   \\answer (Siehe Text.)
\ecode
und das dreimal. Die vierte \cs{answer} schreibt man wie gewohnt.
Man kann aber auch die Kontrolle am Ende der \env{solution} Umgebung
überspringen, indem man in der Lösung (egal wo) den Befehl \cs{NoCheck} schreibt.

Allerdings nimmt \cs{answer} auch ein optionales Argument, das ein Kleinbuchstabe sein muss.
In diesem Fall z.B.~brauchen wir die Lösung ab Teil~d, also machen wir\ldots
\answer[d] \ldots\ und alles ist in Ordnung.

Nein, es gibt keinen Automatismus: man muss selbst zählen! Es gibt zu viele
(unsinnige) Kombinationen, dass alle Fälle abgedeckt werden könnten.
\end{solution}

\clearpage
\exercisesymbol{\onlytikz}
\exercise[!! Experimentell !!]
\addbackgroundgrid
Es gibt ein böses, böses Makro \cs{addbackgroundgrid}, das ein Gitter auf der
\emph{ganzen} Seite druckt. Das Makro nimmt auch ein optionales Argument,
das als Option an \TikZ\ weitergegeben wird, d.h.
\bcode
   \\addbackgroundgrid[\param{\TikZ\ Optionen}]
\ecode
fügt den folgenden Code zum Background hinzu
\bcode
   \\tikz[remember picture,overlay]\{\%\newline
   \null\kern3em\\draw[gray,step=5mm,\param{\TikZ\ Optionen}]\newline
   \null\kern3em(current page.south west)grid(current page.north east);\newline
   \}
\ecode
Damit kann man z.B.\ Farbe oder Gittergröße ändern. Das Hintergrundgitter bleibt
bis zum Ende der aktuellen \env{worksheet} Umgebung, oder bis das Makro
\cs{removebackgroundgrid} verwendet wird.
\begin{solution}
ICH HABE ES NOCH NICHT GANZ DURCHGETESTET, ALSO MIT VORSICHT GENIESSEN!
\end{solution}

\exercisesymbol{\onlytikz}
\exercise[Vektorsummen]
Das Makro \cs{vecsum} nimmt zwei \TikZ-Koordinaten und bildet die Summe mit Parallelogramm.
Die Syntax ist
\bcode
\\vecsum[\param{\TikZ\ Optionen}](\param{coord1})(\param{coord2})[\param{Einheit}];
\ecode
Wird das letzte optionale Argument nicht angegeben, so sieht man nur die Vektoren. Wird
etwas angegeben (es ist für eine Einheit gedacht), so wird die Länge der Vektoren ausgegeben.
Die \TikZ-Optionen können Verschiebungen und Farben enthalten, aber \emph{keine} Skalierung:
\begin{center}
\begin{tikzpicture}
\vecsum(30:4)(120:3)[N];
\vecsum[xshift=4cm,red](1,1)(-1,1);
\vecsum[xshift=3cm,yshift=4cm](3,-1)(20:2);
\vecsum[xshift=6cm](10:2)(-25:1.2)[];
\end{tikzpicture}
\end{center}
Will man die Längen ohne Einheiten haben, so muss man ein leeres optionales Argument angeben.

Das Makro \cs{vecsum} besitzt auch eine Variante mit Stern \cs{vecsum*}, die die Summe nur
in den Lösungen drückt:
\begin{center}
\begin{tikzpicture}
\vecsum*(0,1)(30:2);
\end{tikzpicture}
\end{center}

\end{worksheet}

%%%%%%%%%%%%%%%%%%%%%%%%%%%%%%%%%%%%%%%%%%%%%%%%%%%%%%%%%%%%%%%%%%%%%%%%%%%%%%%%%%%%%%%%%%
%%%%%%%%%%%%%%%%%%%%%%%%%%%%%%%%%%%%%%%%%%%%%%%%%%%%%%%%%%%%%%%%%%%%%%%%%%%%%%%%%%%%%%%%%%

\begin{worksheet}[Schülerversuche]

\noindent
\begin{colorminipage}{red!30!white}[t]{\linewidth}
\centerline{\vrule width0pt height2.5ex depth1ex\bfseries !! VORSICHT !! ÄNDERUNG MIT v1.0.0 !!}
\end{colorminipage}

\smallskip\noindent
In früheren Varianten der Klasse gab es eine Umgebung \env{experiment}, die
eigentlich dasselbe wie \env{worksheet} war, nur dass der Befehl \cs{experiment} zur Verfügung
stand. Das habe ich im Grunde nie verwendet, aber eine Kleinigkeit habe ich behalten:
In einer \env{worksheet} Umgebung kann man auch eben einen Versuch mit \cs{experiment}
angeben. Die Syntax ist die gleiche wie bei \cs{exercise} (außer Schwierigkeitsgrad),
und die Zähler sind unabhängig, d.h.~wir können Folgendes haben:

\experiment[Eingagsversuch]
Tu irgendwas.

\exercise[Foo]
Text.

\exercise[Bar]
Text.

\experiment[Noch ein Versüchschen\ldots]
Tu was anderes.

\exercise[Zähler geht weiter]
Die Titel der Aufgaben/Versuche sind standardmäßig in runden Klammern und mit einem Abstand
von \cs{enskip} von der Zahl platziert. Dies kann natürlich geändert werden, und zwar
mit Hilfe des Makros
\bcode
   \\SetTitleSeparators[\param{Abstand}]\{\param{linkes Zeichen}\}\{\param{rechtes Zeichen}\}
\ecode
Die obligatorischen Argumente sind die Zeichen links und rechts vom Titel (default runde Klammern);
das optionale Argument ist der Abstand zwischen Nummer und Titel (default eben \cs{enskip}). Mit
\bcode%
   \\SetTitleSeparators\{[\}\{\$\\rangle\$\}
\ecode
kriegt man zum Beispiel \SetTitleSeparators{[}{$\rangle$}

\exercise[Sieht blöd aus]
Interessanter wäre zum Beispiel mit
\bcode%
   \\SetTitleSeparators[]\{\\quad-{}-{}-\\quad\}\{\}
\ecode
dann hat man \SetTitleSeparators[]{\quad---\quad}{}
\begin{solution}
Freilich, man könnte auch {\ttfamily\cs{SetTitleSeparators}[\cs{quad}]\{-{}-{}-\cs{quad}\}\{\}} schreiben.
\end{solution}

\experiment[Klar?]
Sowohl \cs{exercise} als auch \cs{experiment} verwenden die gleichen Klammer/Abstände.
\end{worksheet}

%%%%%%%%%%%%%%%%%%%%%%%%%%%%%%%%%%%%%%%%%%%%%%%%%%%%%%%%%%%%%%%%%%%%%%%%%%%%%%%%%%%%%%%%%%
%%%%%%%%%%%%%%%%%%%%%%%%%%%%%%%%%%%%%%%%%%%%%%%%%%%%%%%%%%%%%%%%%%%%%%%%%%%%%%%%%%%%%%%%%%
\twocolumn

\begin{worksheet*}[Zweispaltiges Layout]
Arbeitsblätter haben eine minimale Unterstützung für ein zweispaltiges Format.
Verwendet man das Kernel-Makro \cs{twocolumn}, so wird der Titel trotzdem über
beide Spalten gesetzt.

Die beiden Kernel-Makros \cs{twocolumn} und \cs{onecolumn} starten
allerdings immer eine neue Seite und müssen entsprechend \emph{außerhalb}
einer Umgebung \env{worksheet\optstar} verwendet werden.

Will man \emph{innerhalb} eines Arbeitsblattes zwischen ein- und zweispaltigem
Textformat wechseln, so muss man externe Pakete verwenden, wie zum Beispiel
\pkg{multicol}.

\medium\exercise[Foo]
\lipsum[1-3]

\easy\exercise
Zweispaltiges Format ist schwierig. Um Platz für die Schwierigkeitssymbole
zu machen, musste ich \cs{columnsep} ziemlich erhöhen (\opt{24pt} statt der
eher üblichen \opt{10pt}). Das macht engere Spalten und der Blocksatz
wird schwieriger.

\verb=\raggedright= könnte eine gute Idee sein, vor allem mit der deutschen Sprache\ldots

\end{worksheet*}

\onecolumn

%%%%%%%%%%%%%%%%%%%%%%%%%%%%%%%%%%%%%%%%%%%%%%%%%%%%%%%%%%%%%%%%%%%%%%%%%%%%%%%%%%%%%%%%%%
%%%%%%%%%%%%%%%%%%%%%%%%%%%%%%%%%%%%%%%%%%%%%%%%%%%%%%%%%%%%%%%%%%%%%%%%%%%%%%%%%%%%%%%%%%

\begin{print2}[Titel, wenn man will]
Die Umgebung \env{print2} ist nützlich, um etwas doppelt auf der Seite zu drucken,
das man den Schülern ausgeben kann, z.B.\ kurze Texte mit Bildern, Aufgabenstellungen
oder was auch immer:
\begin{center}
\includegraphics[height=2cm]{example-image}
\end{center}
Der Titel kann als optionaler Parameter angegeben werden.
Standardmäßig ist der horizontale Rand 2\,cm und der vertikale Rand 1,5\,cm. Die Ränder
können durch Optionen geändert werden, z.B.
\bcode
\\begin\{print2\}[hmargin=3cm,vmargin=1cm]
\ecode
Die Option \opt{margin=\param{Länge}} verwendet den gleichen Wert für beide Richtungen.
Will man diese Optionen angeben, so muss der Titel natürlich auch mit Hilfe von \opt{title=\param{Titel}}
angegeben werden. Mit der Option \opt{fontsize=\param{Schriftgröße}} kann man die Schriftgröße
andern.

Die Umgebung \env{print2+} funktioniert ähnlich wie \env{print2}, aber man
kann auch eine Lösung dazu schreiben. Dann wird erst die Variante für Schüler (doppelt)
gedruckt, und dann die Variante mit Lösung (natürlich nur einmal).

Die Umgebung \env{print2-} ist dagegen leicht anders: Sie druckt \emph{auf dieselbe} Seite
ihren Inhalt, einmal mit und einmal ohne Lösung. Ehrlich gesagt weiß ich nicht mehr, wozu
ich sie gebraucht habe\ldots

\end{print2}

%%%%%%%%%%%%%%%%%%%%%%%%%%%%%%%%%%%%%%%%%%%%%%%%%%%%%%%%%%%%%%%%%%%%%%%%%%%%%%%%%%%%%%%%%%
%%%%%%%%%%%%%%%%%%%%%%%%%%%%%%%%%%%%%%%%%%%%%%%%%%%%%%%%%%%%%%%%%%%%%%%%%%%%%%%%%%%%%%%%%%

\begin{print3}
Die Umgebung \env{print3} ist ähnlich wie \env{print2}, nur dass sie
selbstverständlich ihren Inhalt dreimal druckt. Es gibt auch eine Variante \env{print3+},
die danach auch den Text mit Lösung druckt. Auch hier kann man Aufgaben setzen:

\medium\exercise[Irgendwas]
Text text
\begin{center}
\includegraphics[width=.1\linewidth]{example-image-a}
\end{center}
Die zwei horizontalen Linien sollen helfen, das Blatt zu schneiden.

Analog gibt es die Umgebungen \env{print4} und \env{print4+}: Sie machen genau, was
ihr Name nahelegt.

Vorsicht: Verwendet man (wie hier) \str\exercise\ in einer \env{printX} Umgebung und wird
\pkg{hyperref} geladen, so hat man Warnungen über uneindeutige Bookmarks. Ich habe keine
Ahnung, wie man das vermeiden kann, aber das ist kein großes Problem. (Nur eine nervige
Warnung halt.)
\end{print3}

%%%%%%%%%%%%%%%%%%%%%%%%%%%%%%%%%%%%%%%%%%%%%%%%%%%%%%%%%%%%%%%%%%%%%%%%%%%%%%%%%%%%%%%%%%
%%%%%%%%%%%%%%%%%%%%%%%%%%%%%%%%%%%%%%%%%%%%%%%%%%%%%%%%%%%%%%%%%%%%%%%%%%%%%%%%%%%%%%%%%%

\begin{cluecards}% standardmäßig 2x3 auf einem Blatt
\cluecard{Vorne 1}{%
   Hinten 1.\\
   Damit kann man Lösungskärtchen drucken. Das erste Argument von \cs{cluecard}
   wird vorne gedruckt (mit \cs{Large}\cs{bfseries}\cs{centering}; siehe unten, wie man das ändern kann).
   Das zweite Argument kommt hinten.
   \[a=b\]
   Gleichungen und Bilder \includegraphics[width=2cm]{example-image} sind unproblematisch.}
\cluecard{Vorne 2}{Hinten 2\\\onlytikz\\Standardmäßig sind 2x3 Kärtchen auf einem Blatt.}
\cluecard{Vorne 3}{Hinten 3\\Damit das mit vorne/hinten gut funktioniert, muss der Drucker natürlich mitmachen\ldots\ das ist leider nicht unter meiner Kontrolle.}
\cluecard{Vorne 4}{Hinten 4\\Die Standardgeometrie kann geändert werden: mit \cs{begin\{cluecards\}[NxM]}
   werden \opt{N} Spalten und \opt{M} Zeilen erzeugt. Siehe das nächste Beispiel.}
\cluecard{Vorne 5}{Hinten 5\\Auch hier kann man eine Key/Value-Sytax verwenden. \cs{begin\{cluecards\}[3x3]}
   ist dasselbe wie \cs{begin\{cluecards\}[layout=3x3]}.}
\cluecard{Vorne 6}{Hinten 6\\Darüber hinaus kann man die Schrift ändern mit \opt{front=...} (default \cs{Large}\cs{bfseries}\cs{centering})
und \opt{back=...} (default leer).}
\end{cluecards}

%%%%%%%%%%%%%%%%%%%%%%%%%%%%%%%%%%%%%%%%%%%%%%%%%%%%%%%%%%%%%%%%%%%%%%%%%%%%%%%%%%%%%%%%%%
%%%%%%%%%%%%%%%%%%%%%%%%%%%%%%%%%%%%%%%%%%%%%%%%%%%%%%%%%%%%%%%%%%%%%%%%%%%%%%%%%%%%%%%%%%

\begin{cluecards}[layout=3x3,front=\itshape\large,back={\color{red}}] % die "Geometrie" kann geändert werden
\cluecard{Kleiner Hinweis zu A1}{Ein kleiner Hinweis für A1}%
\cluecard{Großer Hinweis zu A1}{Ein großer Hinweis für A2}
\cluecard{Lösung zu A1}{Lösung zu A1}
\cluecard{Kleiner Hinweis zu A2}{Ein kleiner Hinweis für A2}
\cluecard{Großer Hinweis zu A2}{Ein großer Hinweis für A2}
\cluecard{Lösung zu A2}{Die Lösung für A2}
\cluecard{Kleiner Hinweis zu A3}{Ein kleiner Hinweis für A3}
\cluecard{Großer Hinweis zu A3}{Ein großer Hinweis für A3}
\cluecard{Lösung zu A3}{Die Lösung für A3\\Man muss selbst zählen. Hat man zu viele \cs{cluecard}s, so werden die Überzählige ignoriert.}
\cluecard{VORNE}{HINTEN, aber eigentlich nicht, weil es zu viel ist...}
\end{cluecards}

%%%%%%%%%%%%%%%%%%%%%%%%%%%%%%%%%%%%%%%%%%%%%%%%%%%%%%%%%%%%%%%%%%%%%%%%%%%%%%%%%%%%%%%%%%
%%%%%%%%%%%%%%%%%%%%%%%%%%%%%%%%%%%%%%%%%%%%%%%%%%%%%%%%%%%%%%%%%%%%%%%%%%%%%%%%%%%%%%%%%%

\begin{test}[2,M,class=9a,date=32.\ Oktober 2022,v=A,ptspre=>,ptspost=<,logo=example-image]
% Possible keys:
% subject= Unterrichtsfach
% class= Klasse
% date = Datum
% schoolyear = Schuljahr (falls nicht angegeben wird automatisch berechnet)
% v= oder version= oder variant= Version (z.B. A/B, Nachschreiber, oder was auch immer)
% nr= oder number= Nummer der KA
% type= Art (default KA)
% 1,2,3,4 setzen automatisch die Nummer auf 1,2,3,4
% M, Ph sind vordefiniert für Mathe/Physik, d.h. `M' ist dasselbe wie `subject=Mathematik'
% bg,background : kariertes Untergrund
% ptspre,prepts= wird vor der Punktzahl angezeigt
% ptspost,postpts= wird nach der Punktzahl angezeigt
% logo=Dateiname oder leer, dann kein Logo
Eine Klassenarbeit/Test/Wasauchimmer wird in der Umgebung \env{test} gesetzt.
Normalerweise müssen einige Keys gesetzt werden, zum Beispiel:
\bcode
  \\begin\{test\}[%
  \begin{tabular}[t]{@{}l}
  number=2,\\
  subject=Mathematik,\\
  date=32.\ Oktober 2022,\\
  version=A,\\
  ptspre=>,\\
  ptspost=<,\\
  logo=example-image]
  \end{tabular}
\ecode
Hier eine kurze Beschreibung aller Keys:
\begin{compactdesc}
\item[\ttfamily subject=\param{Fach}] legt das Fach fest. Es gibt auch die Optionen \opt{M}, \opt{Ph} und \opt{NwT},
welche äquivalent sind zu \opt{subject=Mathematik}, \opt{subject=Physik} und
\opt{subject=NwT} sind.
\item[\ttfamily number,nr] legt die Nummer fest. Man kann auch nur \opt{1} anstatt von \opt{nr=1}
   schreiben, und das geht bis~\opt{4}. Höhere Zahlen müssen mit \opt{nr=} oder \opt{number=}
   angegeben werden.
\item[\ttfamily date] Selbsterklärend.
\item[\ttfamily class] Auch selbsterklärend.
\item[\ttfamily class*] Weniger selbsterklärend: Diese Datei hat \opt{class=9a} im Quelltext,
   und in der Überschrift findet man ,,Klasse 9a``. Wenn man aber z.B.~eine Oberstufe
   hat, will man das Wort ,,Klasse`` nicht haben. Dann kann man z.B.~\opt{class*=Kursstufe 1}
   verwenden.
\item[\ttfamily schoolyear] Das aktuelle Schuljahr wird automatisch berechnet basierend
   auf dem aktuellen Monat. Alternativ kann man \opt{schoolyear=...} explizit angeben.
\item[\ttfamily type] Art des Tests. Voreingestellt ist ,,KA``, aber mit \opt{type=Test} kann man das ändern.
\item[\ttfamily version/variant/v] Will man verschiedene Varianten haben, so kann man
   e.g.~\opt{v=A} angeben.
\item[\ttfamily ptspre/prepts] Fügt Text vor der Punktzahl ein.
\item[\ttfamily ptspost/postpts] Fügt Text nach der Punktzahl ein.
\item[\ttfamily background/bg=\param{\TikZ\ Optionen}] Fügt das Hintergrundgitter ein. Der Parameter wird weitergegeben, d.h.\
   \opt{bg=red} wird übersetzt in \cs{addbackgroundgrid[red]}. Gibt man nur \opt{bg}
   ohne weitere Angabe, dann kriegt man das ,,normale`` graue Gitter, was von
   \cs{addbackgroundgrid} erzeugt wird.
\item[\ttfamily logo=\param{Dateiname}] Die angegebene Datei wird
   als Logo verwendet, das oben links in der Kopfzeile gedruckt wird. Wird kein Logo
   gewünscht, so einfach nichts angeben\ldots\ Alternativ kann man in der Präambel
   (oder in der Konfigurationsdatei \texttt{drcschool.cfg}, oder in einer
   \texttt{.sco} Konfigurationsdatei) \cs{SetLogo\{Dateiname\}} angeben.
\end{compactdesc}

Alle Befehle, die in einem Arbeitsblatt Verwendung finden, können auch hier
verwendet werden. Es gibt einen Unterschied: das optionale Argument von \cs{exercise}
ist nicht der Titel der Aufgabe, sondern die Punktzahl. Die einzelnen Punktzahlen
werden in die \texttt{aux}-Datei geschrieben, so dass die Gesamtpunktzahl in
der Tabelle automatisch errechnet wird.

\noindent
\begin{colorminipage}{red!30!white}[t]{\linewidth}
Wie bei der Herstellung des Inhaltsverzeichnisses muss \TeX\ nach jeder Änderung
einer Punktzahl mindestens \emph{zweimal} laufen, damit die korrekte Gesamtzahl
errechnet wird!
\end{colorminipage}

\begin{center}
\nocalculator\ 1.\ Teil --- OHNE Taschenrechner
\end{center}
Wie man hier oben sieht, stehen auch die Symbole \cs{calculator} \calculator\
und \cs{nocalculator} \nocalculator\ zur Verfügung.

\medium\exercise[0.5]
Man kann natürlich auch hier die Präfixe \cs{hard}, \cs{medium} und \cs{easy} verwenden.

Die komischen Zeichen ,,>`` und ,,<`` vor/nach der Punktzahl sind hier nur als
Beispiel gegeben. Hier wurde z.B.
\bcode
\\begin\{test\}[...,ptspre=>,ptspost=<,...]
\ecode
verwendet. Meistens nutze ich \opt{ptspre=\{vor.\ \}}
(,,voraussichtlich``, es ist immer gut, sich Spielraum zu lassen\ldots)
\begin{solution}
Text text text text text text text text text text text text text text text text text text
\begin{equation}\label{foo}
a=b
\end{equation}
text text text text text text text text text text text text text text text text text.
Wie man aus Gl.~\eqref{foo} sieht, ist diese Aufgabe sehr einfach.
\end{solution}

\exercise[1]\label{ex}
Die Gesamtpunktzahl wird automatisch gerechnet, stimmt aber erst ab der zweiten
Kompilierung. Dies ist Aufgabe~\ref{ex}.

\begin{center}
\calculator\ 2.\ Teil --- mit Taschenrechner
\end{center}

\exercise[2,5]
Halbe Punkte können sowohl mit Komma als auch mit Punkt als Dezimaltrenner kodiert werden.
\TeX\ ist da schlau genug, beides zu verstehen.
\begin{solution}
Text text text text text text text text text text text text text text text text text text
text text text text text text text text text text text text text text text text text.
\end{solution}

\exercise*[3]
Man kann auch hier Zusatzaufgaben angeben. Deren Punktzahl wird \emph{nicht} zur Gesamtpunktzahl
hinzugefügt.
\begin{solution}
Bin mir nicht sicher, ob das sinnvoll ist, daher könnte es sich mal ändern\ldots
\end{solution}

\exercise[2]
Das Makro \cs{question} kann nur innerhalb der \env{questions\optstar}
oder der Single- bzw.~Multiple-Choice-Umgebungen verwendet werden.
(Außerhalb ergibt es einen Fehler.) Wie bereits gezeigt, druckt die
Sternform den Ihnalt des Makros \cs{starredquestionmark}
links vom Buchstaben. Darüber hinaus kann man (mit oder ohne Stern)
auch eine Teilpunktzahl in eckigen Klammern angeben. Das funktioniert in
\env{questions*}
\begin{questions*}{3}
\question Normale Frage.
\question* Extra Frage.
\question[1] Frage.
\end{questions*}
in der ,,normalen`` \env{questions}
\begin{questions}
\question[0,5] Frage.
\question[2] Frage. Wenn man jetzt \cs{renewcommand\{\textbackslash starredquestionmark\}\{+\}} schreibt,
bekommt man beim nächsten \cs{question*}\renewcommand*{\starredquestionmark}{+}
\question*[1,5] Extra Frage.
\end{questions}
sowie z.B.~in \env{multresponse\optstar} und  \env{multchoice\optstar}
\begin{multresponse}
\question*[1] Frage A
\choice Antwort 1
\choice* Antwort 2
\question Frage B
\choice* Antwort 1
\choice* Antwort 2
\end{multresponse}
\begin{solution}
\answer Bitte Beachten!
\answer Das funktioniert eigentlich auch in Arbeitsblättern, hat dort aber keinen großen Sinn\ldots
\answer Es gibt {\itshape\bfseries absolut keine Kontrolle}, dass die Summe aller Teilpunkte der Gesamtpunktzahl entspricht.
\answer Das ist eigentlich ein ,,Relikt`` meiner Klasse \pkg{unituemnf}, das einfach copy/pasted wurde.
\answer Es fehlen Antworten, aber keine Warnung, weil ich hier \NoCheck\str\NoCheck\ verwendet habe.
\answer Was \emph{möglich} ist, ist aber auch nicht unbedingt \emph{sinnvoll}. Ich finde
horizontale Auf{}listungen mit Punktzahl etwas zu ,,überladen``, aber das ist Geschmackssache.
\end{solution}
\end{test}

%%%%%%%%%%%%%%%%%%%%%%%%%%%%%%%%%%%%%%%%%%%%%%%%%%%%%%%%%%%%%%%%%%%%%%%%%%%%%%%%%%%%%%%%%%
%%%%%%%%%%%%%%%%%%%%%%%%%%%%%%%%%%%%%%%%%%%%%%%%%%%%%%%%%%%%%%%%%%%%%%%%%%%%%%%%%%%%%%%%%%
%%%%%%%%%%%%%%%%%%%%%%%%%%%%%%%%%%%%%%%%%%%%%%%%%%%%%%%%%%%%%%%%%%%%%%%%%%%%%%%%%%%%%%%%%%
%%%%%%%%%%%%%%%%%%%%%%%%%%%%%%%%%%%%%%%%%%%%%%%%%%%%%%%%%%%%%%%%%%%%%%%%%%%%%%%%%%%%%%%%%%

\cleardoublepage
\section{,,Utilities``}
Dieser Abschnitt beschreibt einige Makros, welche die Klasse zur Verfügung stellt.
Da sie nicht notwendigerweise mit Arbeitsblättern zu tun haben bzw.~auch außerhalb der
\env{worksheet} Umgebung funktionieren, werden sie hier vorgestellt.

\subsection{Farbige \env{minipage}}
Die Klasse definiert eine Umgebung \env{colorminipage}, die sich wie \env{minipage}
verhält, ihren Inhalt aber in einer farbigen Box setzt. Es handelt sich im Grunde
um eine Kombination aus \cs{colorbox} und \env{minipage}, aber die Textbreite
wird so gerechnet, dass die angegebene Breite auch den Rand berücksichtigt. Die
Randbreite entspricht ganz gewöhnlich dem Wert von \cs{fboxsep}. Alle optionale
Argumente zu \env{minipage} können wie gewohnt verwendet werden:
\bcode
\leavevmode
\kern-2.5em
\\begin\{colorminipage\}\{\param{Farbe}\}[\param{Ausrichtung}][\param{Höhe}][\param{innere Ausrichtung}]\{\param{Breite}\}
\ecode
Beispiel: mit \verb|\begin{colorminipage}{teal!30!white}[t]{3cm}| hat man:
\begin{colorminipage}{teal!30!white}[t]{3cm}
Hello world, and such, for 'tis better to suffer,
and yet, nothing was meant by this text.\par\xdef\foo{\the\prevdepth}
\end{colorminipage}

\prevdepth\foo
Die Umgebung \env{graybox} ist ein \env{colorminipage}, bei der die Farbe auf
\opt{lightgray} festgelegt wurde, d.h.
\bcode
   \\begin\{graybox\}[\param{Ausr.}][\param{Höhe}][\param{inn.\ Ausr.}]\{\param{Breite}\}
\ecode
ist eine Abkürzung für
\bcode
   \\begin\{colorminipage\}\{lightgray\}[\param{Ausr.}][\param{Höhe}][\param{inn.\ Ausr.}]\{\param{Breite}\}
\ecode

\subsection{Unsichtbares \cs{put}}
Das LaTeX\ Makro \cs{put(\param{x-Shift},\param{y-Shift})\{...\}} platziert ihren
Inhalt verschoben um die Werte \param{x-Shift} und \param{y-Shift}. Die
daraus resultierende Box hat keine Breite, aber sehr wohl Höhe und Tiefe.
Dagegen erzeugt \cs{drcput} eine Box, die keinen Platz in Anspruch nimmt, ansonsten
aber die gleiche Syntax wie \cs{put} hat. Mit einem optionalen Parameter
entscheidet man, ob der Inhalt rechts steht (default, wie bei \cs{put}), oder links,
oder zentriert. Ich mache ein Beispiel und zeichne zuerst einen Pfeil 2\,cm nach rechts
und 2\,ex nach oben als Referenzpunkt. Mit dem Code
\begin{verbatim}
    \rlap{\tikz{\draw[->](0,0)--(2cm,3ex);}}%
    \drcput(2cm,3ex){\textcolor{red}{\LARGE X}}% same as \drcput[r](2cm,3ex)
    \drcput[c](2cm,3ex){\textcolor{blue}{\LARGE X}}%
    \drcput[l](2cm,3ex)[l]{\textcolor{green}{\LARGE X}}%
\end{verbatim}
kriegt man:
\rlap{\tikz{\draw[->](0,0)--(2cm,3ex);}}
\drcput(2cm,3ex){\textcolor{red}{\LARGE X}}
\drcput[c](2cm,3ex){\textcolor{blue}{\LARGE X}}
\drcput[l](2cm,3ex){\textcolor{green}{\LARGE X}}.
Wozu? Na ja, ich benutze es, um z.B.~die Lösungen auf gescannten bzw.~importierten
Arbeitsblätter drauf zu schreiben.

\subsection{Rechte Winkel}
\leavevmode\drcput[l](-1em,3.5ex){\onlytikz}
Mit der \TikZ-Library \opt{angles} (die standardmäßig geladen wird) kann man einen Winkel
zwischen drei Koordinaten plotten. Beispiel: Man definiere drei
Koordinaten $A(0,5|1)$, $B(0|0)$ und $C(1|0)$ und man vergleiche
\begin{center}
\cs{pic[draw]\{angle=C-{}-B-{}-A\};} \textrightarrow\ \tikz{\draw (.5,1)coordinate(A)--(0,0)coordinate(B)--(1,0)coordinate(C) pic[draw]{angle=C--B--A} ; }\qquad
\cs{pic[draw]\{right angle=C-{}-B-{}-A\};} \textrightarrow\ \tikz{\draw(.5,1)coordinate(A)--(0,0)coordinate(B)--(1,0)coordinate(C)pic[draw]{right angle=C--B--A};}\qquad
\cs{pic[draw]\{rightangle=C-{}-B-{}-A\};} \textrightarrow\ \tikz{\draw (.5,1)coordinate(A)--(0,0)coordinate(B)--(1,0)coordinate(C)pic[draw]{rightangle=C--B--A} ; }\qquad
\end{center}
Die ersten zwei Bilder zeigen das Ergebnis von \opt{angle} und \opt{right angle},
welche in \TikZ\ vordefiniert sind. Das dritte Bild dagegen kommt
aus \opt{rightangle} (zusammengeschrieben!). Man beachte, dass man den Punkt in der Mitte
des Winkels immer bekommt, unabhängig davon, ob der Winkel wirklich ein rechter Winkel ist.
(Ich habe absichtlich auf eine entsprechende Kontrolle und Warnung/Fehlermeldung verzichtet:
Manchmal will man einen rechten Winkel in einer 3D-Darstellung markieren, wo der
tatsächlich gezeichnete Winkel nicht wirklich recht ist.)

Als ältere Variante existiert noch das Makro
\bcode
\cs{rightangle}[\param{\TikZ\ Optionen}](\param{coord})\{\param{start angle}\}[\param{radius}]
\ecode
plottet in einem \TikZ-Bild einen rechten Winkel (und zwar immer einen rechten Winkel).
\begin{center}
   \begin{tikzpicture}
   \draw (1,1)--++(30:-2)coordinate(A)--++(120:1) ;
   \rightangle[red](A){30}
   \end{tikzpicture}
\end{center}

\subsection{Einfache Formen von \env{wrapfigure}}
\wrap[r]{\includegraphics[width=2cm]{example-image-16x9}}
Das Makro \cs{wrap[\param{\opt{l} oder \opt{r}}]\{\param{Inhalt}\}} ist eine sehr
abgespeckte Version der vom Paket \pkg{wrapfig} definierten Umgebung \env{wrapfigure}. Standardmäßig
ist das Bild auf der linken Seite: Dies wurde hier mittels \cs{wrap[r]\{...\}} geändert.
Das Makro \cs{wrap} ist dazu gedacht, zu Beginn eines Abschnitts verwendet zu werden,
und daher startet sie immer auch einen neuen Abschnitt.
Wer eine feinere Kontrolle und bessere Ergebnisse möchte, sollte natürlich auf \env{wrapfigure}
zugreifen.

\wraptikz[blue,thick]{\draw[step=.5](0,0)grid(1,1);\draw[->](.2,.8)--++(-40:1);}
\leavevmode\drcput[l](-5em,0pt){\onlytikz}Ähnlich zu \cs{wrap} existiert auch
\bcode
\\wraptikz[\param{\TikZ\ Optionen}][\param{\opt{l} oder \opt{r}}]\{\param{\TikZ\ Code}\}
\ecode
das im Grunde dasselbe wie \cs{wrap} macht aber den
Inhalt gleich in einer \env{tikzpicture} setzt. Die Syntax ist leicht anders, denn es
gibt nun \emph{zwei} optionale Argumente: erst die \TikZ-Optionen und dann \opt{l} (default)
oder \opt{r}.

Vorsicht! Bei \cs{wrap} und \cs{wraptikz}  soll beachtet werden,
dass sie nicht gut reagieren, wenn in ihrer Nähe ein neuer Abschnitt begonnen wird oder gar
ein Seitenumbruch stattfindet. Dann wird das Ergebnis äußerst enttäuschend sein\ldots

%%%%%%%%%%%%%%%%%%%%%%%%%%%%%%%%%%%%%%%%%%%%%%%%%%%%%%%%%%%%%%%%%%%%%%%%%%%%%%%%%%%%%%%%%%
%%%%%%%%%%%%%%%%%%%%%%%%%%%%%%%%%%%%%%%%%%%%%%%%%%%%%%%%%%%%%%%%%%%%%%%%%%%%%%%%%%%%%%%%%%
%%%%%%%%%%%%%%%%%%%%%%%%%%%%%%%%%%%%%%%%%%%%%%%%%%%%%%%%%%%%%%%%%%%%%%%%%%%%%%%%%%%%%%%%%%
%%%%%%%%%%%%%%%%%%%%%%%%%%%%%%%%%%%%%%%%%%%%%%%%%%%%%%%%%%%%%%%%%%%%%%%%%%%%%%%%%%%%%%%%%%

\section{Ausfüllbare PDFs \texorpdfstring{\colorbox{green!30!white}}{}{(NEU! v1.0.0)}}\label{sec:hy}
%\textbf{Achtung! Noch nicht ganz gründlich getestet! (Mangels Versuchskaninchen\ldots) Rückmeldungen willkommen.}
%\vskip1ex\noindent
Lädt man die Klasse mit der Option \opt{hyperworksheet}, so steht auch eine gleichnamige
Umgebung zur Verfügung (sowie eine Sternform dazu). Diese verhält sich im großen und ganzen
wie die normale \env{worksheet}, erzeugt aber eine \emph{ausfüllbare} PDF-Datei, d.h.~mit
Feldern, die mit einem geeigneten PDF-Viewer ausgefüllt werden können.

Damit dies erfolgen kann, muss natürlich das Paket \pkg{hyperref} geladen werden.
Da dies einen ziemlich großen Einschnitt in die Funktion verschiedener Makros und Umgebungen
darstellt, ist dies keine Defaulteinstellung sondern muss eben extra mit Option deklariert
werden.

Wenn jemand unbedingt denkt, das \emph{muss} ihr/sein Standard sein, und keine Lust hat,
die Option \opt{hyperworksheet} jedes Mal zu schreiben, dann kann man
\begin{verbatim}
    \Hyperworksheet
\end{verbatim}
in der Hauptkonfigurationsdatei \opt{drcschool.cfg} schreiben.

\begin{description}
\item[Bemerkung:]
Es gibt einige wenige Pakete, die \emph{nach} \pkg{hyperref} geladen werden müssen:
diese können getrost in der Präambel geladen werden. Wenn jemand allerdings ein Paket
will, das vor \pkg{hyperref} geladen werden muss, so soll dies entweder noch vor
\cs{documentclass} erfolgen, oder in einer Hauptkonfigurationsdatei.
\end{description}

Das folgende Beispiel beschreibt, was darin möglich ist:

%%%%%%%%
\Form  %  < --- IHR BRAUCHT DIES, DAMIT ES FUNKTIONIERT!!
%%%%%%%%
\begin{hyperworksheet}[Mit Vorsicht genießen!]

\begin{center}\fboxsep5pt
\begin{colorminipage}{red!30!white}{.7\linewidth}
Ob und wie genau PDF-Forms funktionieren (oder nicht), hängt \emph{sehr} vom
verwendeten PDF-Viewer ab! Wenn etwas nicht läuft, dann nicht sofort mit den
Schülern/Studenten schimpfen! Vielleicht trifft sie doch (ausnahmsweise) keine
Schuld\ldots
\end{colorminipage}
\end{center}

\noindent
Zuerst muss im Code vor der Umgebung \env{hyperworksheet} das Makro \cs{Form}
aufgerufen werden. Alternativ, kann man das Arbeitsblatt zwischen
\cs{begin\{Form\}} und \cs{end\{Form\}} setzen. Pro Datei kann man nur \emph{einmal}
\cs{Form} (bzw.~eine \env{Form} Umgebung) verwenden. (Für Details sei auf die Dokumentation
von \pkg{hyperref} hingewiesen.)

Die \env{hyperworksheet} Umgebung akzeptiert die gleichen Optionen von \env{worksheet}.
Einige Makros/Umgebungen bekommen eine besondere Bedeutung:

\exercise[Lückentexte]
\begin{cloze}
Lückentexte in der Umgebung \env{cloze} funktionieren genau so: Das Makro \cs{fillhere}
gibt wie gewohnt ihren unterstrichenen \fillhere{Inhalt} in der Variante \fillhere{mit} Lösung,
erzeugt aber eine \fillhere{ausfüllbare} Box in der Variante für Schüler.
Das optionale Argument funktioniert normal: siehe \fillhere{hier}, \fillhere[3]{hier},
\fillhere[3cm]{hier}.

Was NICHT funktioniert ist die Sternform, bzw.~sie funktioniert, macht aber das gleiche
(also keine Linie bis zum \fillhere*{Zeilenende}).
\end{cloze}

\exercise[Wahr/falsch Tabelle]
Die Umgebung \env{TF} kann auch in der PDF angekreuzt werden.
\begin{TF}[.8\linewidth]
Aussage 1 \true
\midrule
Aussage 2, die sehr sehr lang ist, so dass der Text mehr als eine Zeile braucht, um gesetzt zu werden \false
\midrule
Aussage 3 \false
\end{TF}

\exercise[Grids]
Das Makro \cs{grid} ergibt in der Lösung das übliche Ergebnis, erzeugt aber in der Schülervariante
eine mehrzeilig auf{}füllbare Box.

\grid(*,2){Denken ist überbewertet, daher versuche ich, es zu unterlassen. Ach ja, in der füllbaren Variante gibt es keinen
Unterschied zwischen \texttt{*} und \texttt{+} als erstes Argument.}

\noindent
Ein Wort der Vorsicht: Die Boxen in den beiden Fällen sind nicht \emph{genau}
gleich groß, so leider werden
sich (vor Allem vertikale) Abstände (und schlimmstenfalls auch Seitenumbrüche) i.A.~unterscheiden
in den Varianten mit/ohne Lösung.

\easy\exercise[Multiple Response]
Die Umgebung \env{multresponse} funktioniert genauso wie im normalen Fall.
\begin{multresponse}
\question Bei Phänomen A gilt\ldots
\choice* bar bar bar bar bar bar bar bar bar bar bar bar bar bar bar bar bar bar
\choice* baz baz baz baz baz baz baz baz baz baz baz baz baz baz baz baz baz baz
\question Bei Phänomen B gilt\ldots
\choice* test test test test test test test test test test test test test test
\choice yawn yawn yawn yawn yawn yawn yawn yawn yawn yawn yawn yawn yawn yawn
\end{multresponse}

\hard\exercise[Multiple Choice]
(Die Schwierigkeit bezieht sich auf \TeX, nicht auf den Inhalt\ldots)\\
Die Umgebung \env{multchoice} funktioniert
mehr oder weniger auch, aber nicht jeder PDF-Viewer schafft es, sie
richtig darzustellen.
\begin{multchoice*}{3}
\question Frage A
\choice foo
\choice* bar
\choice baz
\question Frage B
\choice foo
\choice bar
\choice* baz
\question Frage C
\choice* foo
\choice bar
\choice baz
\end{multchoice*}
Die Idee einer \env{multchoice} Umgebung ist, dass nur eine Antwort richtig
ist. Wenn man dann einen anderen Feld anklickt, so sollte ein zuerst angeklickter
Radiobutton (aus derselben Frage) ,,weggehen``. Nun, das funktioniert nicht mit
allen Viewern: Auf meinem Windows~10 Rechner hat es funktioniert nur mit den
eingebauten Viewern von Chrome, Edge und Thunderbird. Es hat nicht funktioniert
mit PDF-XChange~Viewer und Okular. Adobe Acrobat Reader habe ich nur auf Handy
und Tablet (beides Android) und er hat dort versagt. (Fairerweise muss ich sagen,
dass \emph{alle} PDF-Readers auf meinem Handy und Tablet versagt haben.)

Eine Wahr-Falsch-Tabelle ist eigentlich also eine sinnvolle (und rechtlich sicherere)
Alternative. Wer sich mit \pkg{hyperref} auskennt, kann natürlich mit \cs{ChoiceMenu}
spielen.

\exercise[Lückentexte als graue Boxen]

Das Makro \cs{fillbox} funktioniert auch: Im Textmodus ist es natürlich genau dasselbe
wie \cs{fillhere}. Im Mathe-Modus skaliert er auch wie gewohnt, aber das Ergebnis ist
noch nicht gaaanz schön, aber es geht:
\[
3^{\fillbox{2}} = 9, \qquad \fillbox{49}^{1/2}=7 .
\]

\hard\exercise[Schlussbemerkung]
Ich habe gerade keinen Ansatz für \env{matching}. Dafür kann man im Grunde eine Tabelle
mit \cs{fillhere} oder \cs{fillbox} aufsetzen.

Wenn man das Blatt online verteilen will, ist es natürlich wesentlich sinnvoller,
die Sternform der Umgebung, \env{hyperworksheet}, zu verwenden. Nutzt man die normale
Variante, so muss man danach noch die Seite(n) ohne Lösung extrahieren.
\end{hyperworksheet}

\end{document}